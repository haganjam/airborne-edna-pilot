% Options for packages loaded elsewhere
\PassOptionsToPackage{unicode}{hyperref}
\PassOptionsToPackage{hyphens}{url}
\PassOptionsToPackage{dvipsnames,svgnames,x11names}{xcolor}
%
\documentclass[
  letterpaper,
  DIV=11,
  numbers=noendperiod]{scrartcl}

\usepackage{amsmath,amssymb}
\usepackage{iftex}
\ifPDFTeX
  \usepackage[T1]{fontenc}
  \usepackage[utf8]{inputenc}
  \usepackage{textcomp} % provide euro and other symbols
\else % if luatex or xetex
  \usepackage{unicode-math}
  \defaultfontfeatures{Scale=MatchLowercase}
  \defaultfontfeatures[\rmfamily]{Ligatures=TeX,Scale=1}
\fi
\usepackage{lmodern}
\ifPDFTeX\else  
    % xetex/luatex font selection
\fi
% Use upquote if available, for straight quotes in verbatim environments
\IfFileExists{upquote.sty}{\usepackage{upquote}}{}
\IfFileExists{microtype.sty}{% use microtype if available
  \usepackage[]{microtype}
  \UseMicrotypeSet[protrusion]{basicmath} % disable protrusion for tt fonts
}{}
\makeatletter
\@ifundefined{KOMAClassName}{% if non-KOMA class
  \IfFileExists{parskip.sty}{%
    \usepackage{parskip}
  }{% else
    \setlength{\parindent}{0pt}
    \setlength{\parskip}{6pt plus 2pt minus 1pt}}
}{% if KOMA class
  \KOMAoptions{parskip=half}}
\makeatother
\usepackage{xcolor}
\setlength{\emergencystretch}{3em} % prevent overfull lines
\setcounter{secnumdepth}{-\maxdimen} % remove section numbering
% Make \paragraph and \subparagraph free-standing
\makeatletter
\ifx\paragraph\undefined\else
  \let\oldparagraph\paragraph
  \renewcommand{\paragraph}{
    \@ifstar
      \xxxParagraphStar
      \xxxParagraphNoStar
  }
  \newcommand{\xxxParagraphStar}[1]{\oldparagraph*{#1}\mbox{}}
  \newcommand{\xxxParagraphNoStar}[1]{\oldparagraph{#1}\mbox{}}
\fi
\ifx\subparagraph\undefined\else
  \let\oldsubparagraph\subparagraph
  \renewcommand{\subparagraph}{
    \@ifstar
      \xxxSubParagraphStar
      \xxxSubParagraphNoStar
  }
  \newcommand{\xxxSubParagraphStar}[1]{\oldsubparagraph*{#1}\mbox{}}
  \newcommand{\xxxSubParagraphNoStar}[1]{\oldsubparagraph{#1}\mbox{}}
\fi
\makeatother


\providecommand{\tightlist}{%
  \setlength{\itemsep}{0pt}\setlength{\parskip}{0pt}}\usepackage{longtable,booktabs,array}
\usepackage{calc} % for calculating minipage widths
% Correct order of tables after \paragraph or \subparagraph
\usepackage{etoolbox}
\makeatletter
\patchcmd\longtable{\par}{\if@noskipsec\mbox{}\fi\par}{}{}
\makeatother
% Allow footnotes in longtable head/foot
\IfFileExists{footnotehyper.sty}{\usepackage{footnotehyper}}{\usepackage{footnote}}
\makesavenoteenv{longtable}
\usepackage{graphicx}
\makeatletter
\def\maxwidth{\ifdim\Gin@nat@width>\linewidth\linewidth\else\Gin@nat@width\fi}
\def\maxheight{\ifdim\Gin@nat@height>\textheight\textheight\else\Gin@nat@height\fi}
\makeatother
% Scale images if necessary, so that they will not overflow the page
% margins by default, and it is still possible to overwrite the defaults
% using explicit options in \includegraphics[width, height, ...]{}
\setkeys{Gin}{width=\maxwidth,height=\maxheight,keepaspectratio}
% Set default figure placement to htbp
\makeatletter
\def\fps@figure{htbp}
\makeatother
% definitions for citeproc citations
\NewDocumentCommand\citeproctext{}{}
\NewDocumentCommand\citeproc{mm}{%
  \begingroup\def\citeproctext{#2}\cite{#1}\endgroup}
\makeatletter
 % allow citations to break across lines
 \let\@cite@ofmt\@firstofone
 % avoid brackets around text for \cite:
 \def\@biblabel#1{}
 \def\@cite#1#2{{#1\if@tempswa , #2\fi}}
\makeatother
\newlength{\cslhangindent}
\setlength{\cslhangindent}{1.5em}
\newlength{\csllabelwidth}
\setlength{\csllabelwidth}{3em}
\newenvironment{CSLReferences}[2] % #1 hanging-indent, #2 entry-spacing
 {\begin{list}{}{%
  \setlength{\itemindent}{0pt}
  \setlength{\leftmargin}{0pt}
  \setlength{\parsep}{0pt}
  % turn on hanging indent if param 1 is 1
  \ifodd #1
   \setlength{\leftmargin}{\cslhangindent}
   \setlength{\itemindent}{-1\cslhangindent}
  \fi
  % set entry spacing
  \setlength{\itemsep}{#2\baselineskip}}}
 {\end{list}}
\usepackage{calc}
\newcommand{\CSLBlock}[1]{\hfill\break\parbox[t]{\linewidth}{\strut\ignorespaces#1\strut}}
\newcommand{\CSLLeftMargin}[1]{\parbox[t]{\csllabelwidth}{\strut#1\strut}}
\newcommand{\CSLRightInline}[1]{\parbox[t]{\linewidth - \csllabelwidth}{\strut#1\strut}}
\newcommand{\CSLIndent}[1]{\hspace{\cslhangindent}#1}

\usepackage{lineno}
\usepackage{setspace}
\linenumbers
\doublespacing
\setlength\linenumbersep{12pt}
\renewcommand\linenumberfont{\normalfont\tiny\sffamily}
\KOMAoption{captions}{tableheading}
\makeatletter
\@ifpackageloaded{caption}{}{\usepackage{caption}}
\AtBeginDocument{%
\ifdefined\contentsname
  \renewcommand*\contentsname{Table of contents}
\else
  \newcommand\contentsname{Table of contents}
\fi
\ifdefined\listfigurename
  \renewcommand*\listfigurename{List of Figures}
\else
  \newcommand\listfigurename{List of Figures}
\fi
\ifdefined\listtablename
  \renewcommand*\listtablename{List of Tables}
\else
  \newcommand\listtablename{List of Tables}
\fi
\ifdefined\figurename
  \renewcommand*\figurename{Figure}
\else
  \newcommand\figurename{Figure}
\fi
\ifdefined\tablename
  \renewcommand*\tablename{Table}
\else
  \newcommand\tablename{Table}
\fi
}
\@ifpackageloaded{float}{}{\usepackage{float}}
\floatstyle{ruled}
\@ifundefined{c@chapter}{\newfloat{codelisting}{h}{lop}}{\newfloat{codelisting}{h}{lop}[chapter]}
\floatname{codelisting}{Listing}
\newcommand*\listoflistings{\listof{codelisting}{List of Listings}}
\makeatother
\makeatletter
\makeatother
\makeatletter
\@ifpackageloaded{caption}{}{\usepackage{caption}}
\@ifpackageloaded{subcaption}{}{\usepackage{subcaption}}
\makeatother

\ifLuaTeX
  \usepackage{selnolig}  % disable illegal ligatures
\fi
\usepackage{bookmark}

\IfFileExists{xurl.sty}{\usepackage{xurl}}{} % add URL line breaks if available
\urlstyle{same} % disable monospaced font for URLs
\hypersetup{
  colorlinks=true,
  linkcolor={blue},
  filecolor={Maroon},
  citecolor={Blue},
  urlcolor={Blue},
  pdfcreator={LaTeX via pandoc}}


\author{}
\date{}

\begin{document}


\textbf{Manuscript type:} Article

\textbf{Title:} Quantifying how biodiversity affects ecosystem
functioning across space and time in natural marine ecosystems

\textbf{Authors:} James G. Hagan\(^{1, 2, 3}\)*, Benedikt
Schrofner-Brunner\(^{2, 3}\) and Lars Gamfeldt\(^{2, 3, 4}\)

\begin{enumerate}
\def\labelenumi{\arabic{enumi}.}
\tightlist
\item
  Community Ecology Lab, Department of Biology, Vrije Universiteit
  Brussel (VUB), Pleinlaan 2, 1050 Brussels, Belgium\\
\item
  Department of Marine Sciences, University of Gothenburg, Box 461,
  SE-40530, Gothenburg, Sweden\\
\item
  Gothenburg Global Biodiversity Centre, Box 461, SE-40530, Gothenburg,
  Sweden\\
\item
  Centre for Sea and Society, Box 260, SE-40530, Gothenburg, Sweden
\end{enumerate}

\textbf{Corresponding author:} James G. Hagan (james\_hagan@outlook.com)

\textbf{Data availability:} Raw data for case study 1 are available on
\href{https://github.com/haganjam/BEF_quant_scale}{Github} and will be
archived upon publication. Raw data for case study 2 are archived on
\href{https://researchbox.org/843&PEER_REVIEW_passcode=GLGJFF}{ResearchBox}
and will be public upon publication.

\textbf{Code availability:} All code used to perform the analysis can be
found on \href{https://github.com/haganjam/BEF_quant_scale}{Github} and
will be archived on Zenodo upon publication.

\textbf{Key words:} biodiversity, ecosystem function, scale,
semi-natural ecosystem, statistical partition

\subsection{Abstract}\label{abstract}

Most empirical work investigating the importance of biodiversity for
ecosystem functioning has taken place on small spatial and temporal
scales. Therefore, we need to improve our understanding at larger
spatial and temporal scales especially if we want biodiversity-ecosystem
functioning science to be relevant for policy and management. We built a
methodological pipeline around a rigorously derived statistical
partition that allowed us to distinguish the various ways in which
biodiversity can affect functioning in two natural marine ecosystems
(intertidal rockpool macroalgae and marine fouling communities). This
partitioning considers both local species interactions and species'
asynchronous responses to varying environments over space and time. We
found that a combination of local-scale species interactions,
local-scale dominance by high functioning species, and spatial niche
partitioning contributed to a positive effect of biodiversity on
ecosystem functioning in these ecosystems. Moreover, the importance of
spatial niche partitioning tended to strengthen with increasing spatial
environmental heterogeneity. Our approach allowed us to directly
quantify the complex ways in which biodiversity can affect ecosystem
functioning and highlights the importance of taking a multi-scale
perspective in biodiversity-ecosystem functioning research.

\subsection{Introduction}\label{introduction}

Concerns over declines in biodiversity in the 1990s prompted ecologists
to study how biodiversity loss may affect the functioning of ecosystems
(deLaplante and Picasso 2011; Frank 2022). Since then, substantial
experimental work has shown that biodiversity is important for
sustaining ecosystem functioning (Cardinale et al. 2012; Tilman, Isbell,
and Cowles 2014). The bulk of these experiments have been performed at
small spatial scales, and researchers have intentionally worked under
relatively homogeneous environmental conditions (Qiu and Cardinale 2020;
Gonzalez et al. 2020; Gamfeldt et al. 2023; Hagan 2024). However,
environmental heterogeneity, both in space and time, is an important
mediator of species coexistence (Hart, Usinowicz, and Levine 2017), and
biodiversity is often changing at multiple scales (McGill et al. 2015).
Thus, there is a need to improve our understanding of the
biodiversity-ecosystem functioning relationship at large spatial and
temporal scales in natural ecosystems (Gonzalez et al. 2020; Isbell et
al. 2017). This is especially important if we want
biodiversity-functioning science to be relevant for policy and
management (Isbell et al. 2017; Le Provost et al. 2023).

At local scales, changes in biodiversity are unambiguously considered to
affect functioning if mixtures of species have higher (or lower)
functioning than expected (Petchey 2003). This expectation is based on
each species' functioning in monoculture (i.e.~without local species
interactions), and the expected functioning of each species in mixture
(referred to as expected relative yield, \(RY_E\)). The deviation of a
mixture from this null expectation is called a net biodiversity effect
and quantifies the magnitude of the effect of biodiversity on ecosystem
function (Loreau and Hector 2001). Using a classic statistical partition
(Loreau and Hector 2001), it was shown that there are at least two sets
of mechanisms that can lead to net biodiversity effects (but see Fox
2005). The first is referred to as complementarity. Positive
complementarity occurs when there is niche partitioning or positive
interactions among co-occurring species outweigh any negative
interactions (Cardinale 2011; Williams et al. 2017). The second is
called selection, which is positive when species with high levels of
ecosystem function in monoculture dominate functioning in species
mixtures (Creed et al. 2009; Spaak and De Laender 2021). Understanding
complementarity and selection is important because their relative
strength determines how we interpret the way biodiversity affects
functioning. Specifically, complementarity is maximised when species
coexist and/or interact positively. This implies that many species are
required to maintain high levels of ecosystem functioning at a local
scale. In contrast, selection is maximised when one or a few high
functioning species dominate functioning, which implies that only one or
a few species are required to maintain high levels of ecosystem
functioning.

Experimental work at local scales has shown that complementarity and
selection are frequently of similar magnitude (Cardinale et al. 2007;
Cardinale 2011) (but see Hong et al. 2022). Moreover, in experiments,
the highest functioning species in monoculture often has higher
functioning than the most diverse mixture of species (Gamfeldt et al.
2023, 2015; Cardinale 2011), and in many natural systems, a few dominant
species tend to drive ecosystem functioning (Smith and Knapp 2003; Smith
et al. 2020; Lisner et al. 2023). Even though local functioning might be
largely controlled by a few dominant species, this does not mean that
biodiversity is not important. Rather, if the identity of the dominant
species varies across space and time, biodiversity may still be
important at larger scales. This idea has been formalised as the
insurance hypothesis for the effect of biodiversity on ecosystem
functioning (reviewed in Loreau et al. 2021). The insurance hypothesis
states that, when species respond asynchronously to environmental
variation in space and time and dominate places and times where/when
they are highest functioning in monoculture, biodiversity increases both
the magnitude and stability of ecosystem functioning at larger scales
(Yachi and Loreau 1999; Loreau, Mouquet, and Gonzalez 2003). For
example, a temporal insurance effect occurs when species respond
asynchronously to temporal environmental variation and dominate during
the times when they are highest functioning (Yachi and Loreau 1999).

Attempts to quantify the effects of biodiversity on ecosystem
functioning at large scales have generally focused on examining whether
the slope of the relationship between biodiversity and ecosystem
functioning changes with increasing spatial/temporal extents or grains
(where extent and grain are defined \emph{sensu} Chase and Knight 2013).
Theoretical models investigating this have typically found that the
slope can indeed change with increasing grain (Thompson et al. 2018,
2021) and extent (Barry et al. 2021). Empirical studies that correlated
biodiversity with measures of ecosystem functioning at different spatial
extents and grains in natural systems (Craven et al. 2020; Chisholm et
al. 2013) reached similar conclusions. One issue with these studies is
that the change in the slope of the relationship between biodiversity
and functioning is sensitive to details such as whether functioning is
averaged or summed within a given spatial unit (Barry et al. 2021), or
how spatial units are aggregated to increase scale (compare Thompson et
al. 2018, 2021). The focus on the slope means that neither net
biodiversity effects nor insurance effects have been directly calculated
or compared across spatial and temporal scales in natural ecosystems.

\begin{figure}

\centering{

\includegraphics{figures/main_fig_1.png}

}

\caption{\label{fig-m1}Overview of the additive statistical partition of
biodiversity effects across spatial and temporal scales. \textbf{(a)}
How the net biodiversity effect (NBE) is calculated and additively
partitioned into total complementarity (TC) and total selection (TS).
Total selection is then further partitioned into five terms that
describe how biodiversity can affect function across times and places:
average selection (AS), temporal insurance (TI), spatial insurance (SI),
spatio-temporal insurance (ST) and a residual term called non-random
overyielding (NO). Total insurance (IT) is the sum of AS, TI, SI and ST
(see Table~\ref{tbl-table-1} for abbreviations). \textbf{(b)}
Hypothetical examples with two species, two times and two places to give
an intuition of what these biodiversity effects measure. For example,
total complementarity quantifies the part of the net biodiversity effect
that is due to local-scale interactions among species whilst spatial
insurance quantifies the extent to which species that are high
functioning in monoculture at a particular place tend to dominate the
mixtures in those places (see Appendix 1, Table S 1 for ecological
explanations of the different terms).}

\end{figure}%

\begin{longtable}[]{@{}
  >{\raggedright\arraybackslash}p{(\columnwidth - 2\tabcolsep) * \real{0.4028}}
  >{\raggedright\arraybackslash}p{(\columnwidth - 2\tabcolsep) * \real{0.5972}}@{}}
\caption{Guide to the symbols used in the equations of the statistical
partition. Symbols correspond to those used in Figure~\ref{fig-m1}, the
\emph{Materials and methods} section, and Appendix
2.}\label{tbl-table-1}\tabularnewline
\toprule\noalign{}
\begin{minipage}[b]{\linewidth}\raggedright
Symbol
\end{minipage} & \begin{minipage}[b]{\linewidth}\raggedright
Explanation
\end{minipage} \\
\midrule\noalign{}
\endfirsthead
\toprule\noalign{}
\begin{minipage}[b]{\linewidth}\raggedright
Symbol
\end{minipage} & \begin{minipage}[b]{\linewidth}\raggedright
Explanation
\end{minipage} \\
\midrule\noalign{}
\endhead
\bottomrule\noalign{}
\endlastfoot
\(P, T, S\) & Number of places (\(k\)), times (\(j\)), and species
(\(i\)) respectively \\
\(M_{ijk}\) & Monoculture function of species \(i\), at place \(k\) and
time \(j\) \\
\(Y_{ijk}\) & Observed function of species \(i\), at place \(k\) and
time \(j\) in mixture \\
\(RY_{E,ijk}\) & Expected initial proportion of species \(i\) at place
\(k\), and time \(j\) in mixture \\
\(RY_{O,ijk} = \frac{Y_{ijk}}{M_{ijk}}\) & Observed relative yield of
species \(i\) at place \(k\) and time \(j\) in mixture \\
\(\Delta RY_{ijk} = RY_{O,ijk} - RY_{E,ijk}\) & Change in relative yield
of species \(i\) at place \(k\) and time \(j\) in mixture \\
\(p_{O,ijk}\) & Observed relative functioning of species \(i\), at place
\(k\) and time \(j\) in mixture \\
\(\Delta RY_{O,ijk} = RY_{O,ijk} - p_{O,ijk}\) & Change in observed
relative yield of species \(i\) at place \(k\) and time \(j\) in
mixture \\
\(\Delta p_{ijk} = p_{O,ijk} - RY_{E,ijk}\) & Change in relative
functioning of species \(i\) at place \(k\) and time \(j\) in mixture \\
\end{longtable}

Recently, an extension of the original complementarity-selection
statistical partition (Loreau and Hector 2001) was developed (Isbell et
al. 2018). The extension quantifies a net biodiversity effect in
different places and times and partitions it into a set of effects that
describe how biodiversity can affect functioning at local and larger
scales of space and time (see Figure~\ref{fig-m1} for an overview, and
Appendix 2 for details). Specifically, total complementarity quantifies
the average effect of local-scale interactions among coexisting (or
co-occurring) species on functioning. For example, if total
complementarity is positive then, on average, interactions between
individuals of different species reduce functioning less than
interactions between individuals of the same species
(Figure~\ref{fig-m1} b). Total selection is divided into non-random
overyielding and total insurance. Non-random overyielding is a residual
term describing the tendency for species that are high functioning in
monoculture to overyield most and is usually negative
(Figure~\ref{fig-m1} b and Appendix 1, Table S 1 for details). Total
insurance, when positive, quantifies the extent to which species that
are high functioning in monoculture tend to dominate mixtures. Dominance
of species in mixtures that are high functioning in monoculture occurs
in four ways (Isbell et al. 2018): average selection, spatial insurance,
temporal insurance and spatio-temporal insurance (Figure~\ref{fig-m1}
a). For example, average selection quantifies part of the net
biodiversity effect that is due to a single high functioning species in
monoculture dominating across all places and times (Figure~\ref{fig-m1}
b), whilst spatial insurance quantifies the part of the net biodiversity
effect that is due to different species dominating mixtures in different
places where they are also high functioning in monoculture
(Figure~\ref{fig-m1} b). Quantifying these effects allows the estimation
and comparison of biodiversity effects on ecosystem functioning through
local-scale interactions, dominance by high functioning species along
with spatial and temporal niche partitioning at large scales. As a
result, this statistical partition has clear advantages over approaches
that examine the slope of the relationship between biodiversity and
functioning.

Despite its potential, this approach has only been applied in the
original paper on data from a controlled grassland biodiversity
experiment (Isbell et al. 2018; Reich et al. 2001) and, more recently,
in a spatial context (Castillioni and Isbell 2023). Here, we fill this
gap by directly quantifying the magnitude of the effect of biodiversity
on ecosystem functioning at large spatial and temporal scales in two
natural marine ecosystems: intertidal rockpool macroalgae (\emph{case
study 1}), and marine fouling communities (\emph{case study 2}). For the
rockpool macroalgae, we created monocultures of four species of
canopy-forming brown macroalgae and a mixture of all four species, which
we sampled three times over a six-month period at two rocky shores
(i.e.~two places) (see Figure~\ref{fig-m2} a and \emph{Materials and
methods}). The marine fouling study included monocultures and mixtures
of five marine sessile operational taxonomic units, from nine different
clusters consisting of three to five places each (39 places in total).
We sampled the fouling communities three times over six weeks (see
Figure~\ref{fig-m2} b and \emph{Materials and methods}). In contrast to
experiments in controlled environments, data from natural systems often
have incomplete data on species in monoculture and their expected
relative yields (\(RY_{E}\)s). Our rockpool study had complete data for
monocultures but lacked information on \(RY_{E}\)s, whereas the fouling
data had both incomplete monoculture data and no information on
\(RY_{E}\)s. Therefore, we developed a methodological pipeline that
combines data imputation techniques for dealing with incomplete
monoculture data along with uncertainty assumptions for the \(RY_{E}\)s
(\emph{Materials and methods} and Appendix 3). Using this pipeline and
applying the partition to two marine ecosystems allowed us to determine
both the magnitude of the net biodiversity effect and how much of it is
due to local-scale interactions compared to insurance effects operating
at larger scales. We show that insurance effects of biodiversity can be
large and positive and tend to increase when we consider larger spatial
scales where spatial environmental heterogeneity is higher. In doing so,
we provide insights into how biodiversity at different spatial and
temporal scales contributes to ecosystem functioning.

\begin{figure}

\centering{

\includegraphics{figures/main_fig_2.png}

}

\caption{\label{fig-m2}Data structure of the marine ecosystems for which
we quantified the biodiversity effects. \textbf{(a)} The rockpool
macroalgae dataset (case study 1) consists of two shores (KS --
Kingsand, CB -- Challaborough) near Plymouth, United Kingdom, for which
the percentage cover of monocultures (\(M\)) of four species
(\emph{Bifurcaria bifurcata}, \emph{Fucus serratus}, \emph{Laminaria
digitata} and \emph{Sargassum muticum}) and a mixture (\(Y\)) of the
four species were measured three times (\(t_{1-3}\)). \textbf{(b)} The
marine fouling communities (case study 2) consist of nine clusters
(cluster F was excluded due to loss of replicates) of between three and
five sites in the Tjärnö archipelago on the Swedish west coast (39
places in total). We measured the dry biomass of mixtures at each of the
39 places on three times (\(t_{1-3}\)). We also gathered 287 direct
measurements of monoculture dry biomass for the five operational
taxonomic units across places and times. We used these monoculture
measurements to impute the remaining 298 required monoculture dry
biomass measurements (\emph{Materials and methods}).}

\end{figure}%

\subsection{Materials and methods}\label{materials-and-methods}

\subsubsection{Overview of the methodological
pipeline}\label{overview-of-the-methodological-pipeline}

The statistical partition we used to calculate the different
biodiversity effects (Figure~\ref{fig-m1} and Appendix 2) was originally
developed for experimental data (Isbell et al. 2018). This is because
the partition requires extensive monoculture (i.e.~monoculture data for
all species in all places and at all times for which mixture data are
available), which can relatively easily be generated by inoculating
species into habitat patches in monoculture (Cardinale 2011; Reich et
al. 2001). In addition, seven of the nine biodiversity effects (net
biodiversity effect, total selection, total insurance, average
selection, spatial insurance, temporal insurance and spatio-temporal
insurance) require knowing species' expected relative yields
(\(RY_{E}\)s) for all species in all places and at all times (see
Appendix 3 for an explanation as to why total complementarity and
non-random overyielding can be calculated without \(RY_{E}\)s). In most
biodiversity-ecosystem functioning experiments, mixtures of species are
inoculated with equal initial relative abundance. For example, in a
two-species mixture at a given place and time, both species would be
initially inoculated with 0.5 relative abundance. These initial relative
abundances are then used as the \(RY_{E}\) (Loreau and Hector 2001).

The problem with data from natural systems is that we usually do not
have monoculture data for all species in all places and at all times.
Rather, we might have patchy, natural monoculture data (Gamfeldt et al.
2013; Jochum et al. 2020) or we might be able to collect some
monoculture data through, for example, species removals (Sears and
Chesson 2007). Moreover, we do not have \(RY_{E}\)s because we do not
know the colonisation history that occurred in a given place over time.
To apply the statistical partition and calculate the different
biodiversity effects on field data from natural systems requires a
methodological pipeline that can solve these issues.

Here, we propose such a pipeline. First, to deal with the problem of
incomplete monoculture data, we use data imputation techniques. If we
are able to gather enough known monoculture (i.e.~naturally occurring
monocultures or artificially created monocultures), we can then use
statistical data imputation methods to obtain a distribution of
estimates for the required missing monoculture data. Obtaining a
distribution of estimates for each missing monoculture is especially
suited to a Bayesian framework which is what we use (see Appendix 4.6
for a detailed description of our method of monoculture imputation,
which we applied in our case study on marine fouling communities).

Second, to deal with not knowing the \(RY_{E}\)s, we assume that all
species are initially present at all times and in all places but that
the exact \(RY_{E}\)s of each species are unknown. We then assume a set
of random \(RY_{E}\)s that we draw from a Dirichlet distribution (a
probability distribution that outputs a simplex i.e.~a set of numbers
that sum to 1) and calculate the different biodiversity effects. We
repeat these calculations 100 times with different sets of \(RY_{E}\)s.
This provides a distribution of possible biodiversity effects, which
includes the uncertainty from not knowing the true \(RY_{E}\)s in each
place and at each time (Appendix 3). On simulated data, the Pearson
correlation coefficient between the true, simulated biodiversity effect
and the mean of the distribution of biodiversity effects obtained by
assuming random sets of \(RY_{E}\)s exceeded 0.94 for the seven
biodiversity effects that are sensitive to \(RY_{E}\)s (Appendix 3,
Figure S 11). Moreover, the 95\% percentile interval of each estimated
biodiversity effect (obtained from calculating these effects 100 times
assuming different RY\_\{E\}s) contained the true, simulated
biodiversity effect in more than 80\% of the simulations for all
biodiversity effects except average selection and spatial insurance
(Appendix 3, Table S 7). But even in those cases, the bias was
relatively small in comparison to the average size of the effect (see
Appendix 3 for further analyses). Thus, on simulated data, this
procedure provides a reliable way to quantify the biodiversity effects
when \(RY_{E}\)s are unknown.

\subsubsection{Case study 1: Rockpool
macroalgae}\label{case-study-1-rockpool-macroalgae}

We used our methodological pipeline to apply the statistical partition
of biodiversity effects to data from two natural marine ecosystems. The
first is a case study based on a removal experiment where communities of
four macroalgal species (\emph{Bifurcaria bifurcata} Ross, \emph{Fucus
serratus} Linnaeus, \emph{Laminaria digitata} (Hudson) Lamouroux, and
\emph{Sargassum muticum} (Yendo Fensholt) were manipulated in marine
rockpools near Plymouth, United Kingdom. The four species were chosen
because they are conspicuous members of the local intertidal flora and
can dominate the cover in tide pools (Lars Gamfeldt, personal
observation). The removal was done on two different shores
(Challaborough, 50° 17' N, 3° 54' W and Kingsand, 50° 20' N, 4° 11' W)
that were relatively environmentally similar (Lars Gamfeldt, personal
observation). In June 2007, the rock pools were manually cleared of all
visible animal and algae biomass. All pools were at the mid-height of
the shores and ranged in size from approximately 0.2 to 1 m in diameter
and 0.1 to 0.4 m in depth. We manipulated the presence and absence of
the four focal species by continuously (monthly throughout the duration
of the experiment) scraping off the holdfasts of all small germlings of
the target species. By preventing the establishment of new recruits, we
created two treatments: \textbf{(i)} monocultures of each focal algal
species and \textbf{(ii)} a mixture of the four focal species. There
were between three and four replicate rockpools each monoculture, and
between six and seven rockpools for the mixture. Treatments were
randomly assigned to rockpools. The percentage cover of algae in the
rockpools as a proxy for biomass was estimated three times: March, July
and September 2008. Algal cover typically correlates strongly with
biomass (Stachowicz, Whitlatch, and Osman 1999; Bracken et al. 2017) but
see (Masterson et al. 2008) for counterexamples. At each site and at
each time, we averaged the total percentage cover across the replicates
for the mixture and the monoculture rockpools. Thus, we have mixture and
monoculture data for two places and three times (Figure~\ref{fig-m2} a).

Given that we had complete monoculture data for this case study, we
simply calculated the different biodiversity effects assuming a set of
100 random \(RY_{E}\)s drawn from the Dirichlet distribution, which left
us with a distribution of 100 biodiversity effects for the net
biodiversity effect, total selection, total insurance, average
selection, spatial insurance, temporal insurance and spatio-temporal
insurance (total complementarity and non-random overyielding are
unaffected by \(RY_{E}\)s, Appendix 3). We parameterised the Dirichlet
distribution with four \(\alpha\)-values equal to three. Exploratory
simulations showed that \(\alpha\)-values equal to three provide a good
spread of possible \(RY_{E}\)s (Appendix 3, Figure S 7). We also
examined the relationship between monoculture percentage cover and
relative abundances across times and places. Because the two different
shores were environmentally similar and the time between measurements
was only between five and six months, we did not expect strong insurance
effects of biodiversity in this dataset.

\subsubsection{Case study 2: Marine fouling
communities}\label{case-study-2-marine-fouling-communities}

The second case study includes data on communities consisting of up to
five species of marine fouling organisms on settling plates. We created
mixtures and monocultures with locally occurring marine fouling
organisms between the 7th of July 2022 and the 5th of September 2022 (60
days) in the archipelago around Tjärnö (58° 52' 30.0 '' N 11° 08' 42.0
'' E) on the Swedish west coast (Appendix 4, Figure S 12b). For this, we
installed 600 PMMA panels (10 x 11 cm, transparent with a removable
black background) at 50 sites in the archipelago. Each site consisted of
12 panels mounted on a frame hanging from a buoy at 3 m or 6 m depth
(Appendix 4, Figure S 12). Of the 12 panels, three were randomly
assigned as mixture panels and nine were assigned as monoculture panels.
The three mixture panels correspond to three temporal replicates that we
sampled destructively at three different times. The 50 sites were
grouped into 10 clusters to represent different seascapes (Appendix 4,
Figure S 12a). We made sure that there was at least 40 m between sites
within a cluster.

When choosing the sites, we aimed to create five environmentally
heterogeneous and five environmentally homogeneous clusters. We did this
using \emph{a priori} geographical data (see Appendix 4.2). After the
experiment, we then directly quantified the level of spatial
environmental heterogeneity in each cluster by calculating multivariate
dispersion on 14 z-score standardised environmental variables: turbidity
(depth of visibility (m), 15 m resolution), relative wave exposure
(unitless, 15 m resolution) from geographic data (Greeve et al. 2023)
along with variables that we measured \emph{in situ} during the
experiment, namely: distance from panel to the seabed (m), panel depth
(m), the average, coefficient of variation, maximum and minimum
temperature over the course of the experiment (ºC) and the average,
coefficient of variation and maximum light level over the course of the
experiment (lux), salinity (ppt), Secchi depth (m) and water movement
(mass reduction g hour-1) (see Appendix 4.3 for details). Multivariate
dispersion was calculated using the betadisper() function from the vegan
package (Oksanen et al. 2024) in R v4.1.2. We then compared multivariate
dispersion between clusters originally designated as heterogeneous and
homogeneous using Welch's two-sample t-test implemented using the
t.test() function in R v4.1.2. We visualised the variation in
environmental heterogeneity between clusters using a Principal
Components Analysis (PCA) based on the same 14 z-score standardised
environmental variables which we implemented using the prcomp() function
in R v4.1.2. This analysis showed that there were considerable
differences in spatial environmental heterogeneity between clusters
(Appendix 1, Figure S 4).

For the mixture panels, organisms were allowed to settle and were left
un-manipulated until sampled. For the monoculture panels, we waited
until an organism had settled and was large enough to be identified and
then removed all other species. This provided patchy monoculture data
for different species at different times and different places. The
organisms were identified to their lowest possible taxonomic resolution.
Based on these identifications, we designated five operational taxonomic
units (OTUs): Barn -- the barnacle, \emph{Amphibalanus improvisus}, Bryo
-- three bryozoa species (\emph{Electra pilosa}, \emph{Membranipora
membranacea}, \emph{Callopora rylandi}), Asci -- the solitary ascidian,
\emph{Ascidiella scabra}, Hydro -- two hydrozoan species (\emph{Clytia}
sp. and \emph{Laomedea flexuosa}), and Ciona -- the solitary ascidian
\emph{Ciona intestinalis}. We hereafter refer to these taxonomic units
as OTUs (reference pictures for the OTUs are
\href{https://researchbox.org/843&PEER_REVIEW_passcode=GLGJFF}{available}).

Monocultures were maintained on days 21, 32 and 42 during the
experiment, and all panels were photographed before manipulation took
place (Camera: Sony a6400 with Sony FE 2.8 / 50mm Macro and a
polarisation filter). We retrieved mixtures and monocultures at three
different time points (day 34, day 47 and day 60) for measurement in the
lab. To avoid edge effects and disturbances from attachment holes39, we
removed the panels' outer margin (1 cm). We photographed each panel on a
light table (daylight Wafer 3), scraped off the organisms, and weighed
them to 0.001 g accuracy (scale: Sartorius BP 110). For mixtures, the
different OTUs were weighed separately. Then, samples were dried at 65
°C in a drying oven for 48 hours and weighed again to obtain dry
biomass. Monoculture cover was determined using imageJ2 (Rueden et al.
2017). We divided monoculture dry biomass by monoculture cover and
multiplied by the total area of the panel to make the biomass
measurements comparable to the unmanipulated mixtures (Appendix 4.4). We
made sure that the panels representing different temporal replicates had
similar species composition by correlating OTU cover values across
time-points (Appendix 4.5).

During the experiment we lost 11 sites (including cluster F which we
removed because only two sites were present at the end of the study)
leaving us with 39 sites out of the original 50. This left us with data
for 39 mixtures sampled at three time points (117 datapoints) and 287
monoculture datapoints. A complete mixture-monoculture dataset for this
experiment would consist of monocultures of all five OTUs for the 39
mixtures at all three time points (i.e.~39 sites × 3 time points × 5
OTUs = 585 monoculture datapoints). We imputed the 298 missing
monoculture dry biomass values using multilevel generalised linear
models (GLM) fit in a Bayesian framework. To do this, we fit seven
multilevel GLMs with the observed monoculture biomass (M) as the
response variables and different combinations of each OTU's biomass in
mixture (Y) along with environmental variation quantified using two
principal components (PC1 and PC2). These two principal components were
based on a PCA of nine environmental variables that were available at
all places and times for the mixtures and monocultures, namely: distance
from panel to the seabed (m), panel depth (m), the average, coefficient
of variation, maximum and minimum temperature over the course of the
experiment (ºC) and the average, coefficient of variation and maximum
light level over the course of the experiment (lux). Together, PC1 and
PC2 explained 74\% of the variation in the nine environmental variables
(Appendix 1, Figure S 5). To account for the positive continuous
distribution of observed monoculture biomass values combined with the
presence of true zeros (i.e.~place-time combinations where a species had
zero monoculture biomass), we used a Log-Normal hurdle model. A
Log-Normal hurdle model is a mixture model where the response variable
is modelled as zero or non-zero (0-1) using a Bernoulli process
(i.e.~logistic regression). Then, if the response variable is non-zero,
it is modelled using a Log-Normal distribution. We present all seven
models along with an intercept-only null model with priors in the
supplementary information (Appendix 4.6). The posterior distributions of
the seven models were estimated using Stan's No-U-Turn Sampler
Hamiltonian Monte Carlo algorithm (https://mc-stan.org) with four
separate chains. We implemented this in R v4.1.2 using the rstan package
(2024). We assessed model convergence by inspecting trace plots, the
Gelman-Rubin statistic (R-hat values) and effective samples sizes.

The seven models were ranked using the leave-one-out cross validation
score which we estimated using Pareto Smoothed Importance Sampling
(PSIS) implemented using the loo() function from the loo package Vehtari
et al. (2024) in R v4.1.2. The model with the best leave-one-out cross
validation score (Appendix 1, Table S 3) is presented below (Model 1).
Both the generalised linear models in Model 1 (i.e.~the Bernoulli model
and the Log-Normal model) were fit as multilevel models where the
parameters (i.e.~intercepts and slopes) were fit as correlated random
effects. Using 1000 samples from the posterior distribution of Model 1
(below), we imputed the dry biomass of missing monoculture values
leaving us with 1000 separate estimates of the dry biomass for each
missing monoculture (see Appendix 4.6 for an analysis of the quality of
the imputations).

\[
\begin{aligned}
w_i &\sim \text{Bernoulli}(p_i) \\
w_i &= 
\begin{cases}
0 & \text{if } M_i = 0 \\
1 & \text{if } M_i > 0
\end{cases} \\
\text{logit}(p_i) &= \theta_{s[i]} + \delta_{1_{s[i]}} Y_i \\
\theta_j &= \bar{\theta} + \nu_1 \\
\delta_{1_j} &= \bar{\delta}_1 + \nu_2 \\
\nu &= \left( \text{diag}(\tau) \times \text{cholesky}(R) \times V^\top \right) \\
V_{k,j} &\sim \text{Normal}(0, 1) \\
\bar{\theta}, \bar{\delta}_1 &\sim \text{Normal}(0, 1.5) \\
\tau_\theta, \tau_\delta &\sim \text{Exponential}(3) \\
R_{k,k} &\sim \text{LKJcorr}(2) \\
(M_i \mid w_i = 1) &\sim \text{LogNormal}(u_i, \varepsilon) \\
u_i &= \alpha_{s[i]} + \beta_{1_{s[i]}} Y_i + \beta_{2_{s[i]}} PC1_i \\
\alpha_j &= \bar{\alpha} + z_1 \\
\beta_{1_j} &= \bar{\beta}_1 + z_2 \\
\beta_{2_j} &= \bar{\beta}_2 + z_3 \\
z &= \left( \text{diag}(\sigma) \times \text{cholesky}(L) \times Z^\top \right) \\
\varepsilon &\sim \text{Exponential}(5) \\
Z_{l,j} &\sim \text{Normal}(0, 1) \\
\bar{\alpha}, \bar{\beta}_1, \bar{\beta}_2 &\sim \text{Normal}(0, 1) \\
\sigma_\alpha, \sigma_{\beta_1}, \sigma_{\beta_2} &\sim \text{Exponential}(3) \\
L_{l,l} &\sim \text{LKJcorr}(2)
\end{aligned} \tag{Model 1}
\]

Using all 1000 estimates of monoculture dry biomass for the missing 298
datapoints and assuming a set of a 100 random \(RY_{E}\)s drawn a from
the Dirichlet distribution, we calculated the different biodiversity
effects. For total complementarity and non-random overyielding, we
obtained 1000 different estimates (reflecting the uncertainty in the
monoculture imputations). For the net biodiversity effect, total
selection, total insurance, average selection, spatial insurance,
temporal insurance and spatio-temporal insurance, we obtained 100 000
different estimates reflecting the combined uncertainty in the
monoculture imputations and the \(RY_{E}\)s.

To obtain a pooled mean across the nine clusters for each of the
biodiversity effects, we used a random-effect meta-analysis model
(Koricheva, Gurevitch, and Mengersen 2013). For each biodiversity effect
and each of the nine clusters (i), we calculated the mean and standard
deviation (s) of the distribution of biodiversity effect estimates.
Using the mean and standard deviation of each cluster (s), we fit an
intercept-only random-effects meta-analysis model for each biodiversity
effect as:

\[
\begin{aligned}
\bar{x}_i &\sim \text{Normal}(\theta_i, \sigma) \\
\sigma^2 &= \frac{s^2}{n} \\
\theta_i &\sim \text{Normal}(\mu, \tau)
\end{aligned} \tag{Meta-analysis model}
\]

Where \(n\) is the number of samples (in this case, 1000 or 100 000),
\(\theta_i\) is the true mean of the \(i^{th}\)) function from the
metafor package (Viechtbauer 2010) in R v4.1.2 using restricted maximum
likelihood. We assessed whether the pooled mean (\(\mu\)), which is the
model intercept in this case, differed from zero using the Knapp-Hartung
method which is based on t-tests (Viechtbauer 2010).

Finally, we tested the hypothesis that spatial insurance effects
strengthen with spatial environmental heterogeneity because responses to
the environment are species-specific (or, in this case, OTU-specific).
To do this, we regressed spatial environmental heterogeneity which we
quantified as the multivariate dispersion of each cluster based on 14
environmental variables (Appendix 1, Figure S 4) with all 100 000
estimates of the spatial insurance effect of each cluster. We then
examined the distribution of 100 000 slope estimates for the effect of
multivariate dispersion on the spatial insurance effect. The regressions
were performed using the lm() function in R v4.1.2 which uses ordinary
least squares.

Unless otherwise stated, all analyses were performed in R v4.1.2.
Additional packages used were dplyr (Wickham et al. 2023), tidyr
(Wickham, Vaughan, and Girlich 2024) and readr (Wickham, Hester, and
Bryan 2024) for data handling, ggplot2 (Wickham 2016), ggpubr
(Kassambara 2023) and cowplot (Wilke 2024) for data visualisation,
gtools (Warnes et al. 2023) for the Dirichlet distribution and renv
(Ushey and Wickham 2024) for package version management.

\subsection{Results}\label{results}

\subsubsection{Case study 1: Rockpool
macroalgae}\label{case-study-1-rockpool-macroalgae-1}

The net biodiversity effect in the rockpool macroalgae communities was
large and positive (Mean, Highest Posterior Density Interval
(\(HPDI_{95\%}\)): 126, \(119 - 136\) \% cover). This indicates that
going from one to four species increases cover by more than 100\% across
the three time points and two places, and this effect was dominated by
total complementarity (Figure~\ref{fig-m3} a). Strong total
complementarity indicates that, on average, niche partitioning and/or
positive interspecific interactions have a stronger effect on algal
cover than intraspecific interactions. This is consistent with the
mixtures having higher cover than any of the four monocultures across
all times and places (Appendix 1, Figure S 1a, b).

Total insurance was less than a third of the net biodiversity effect
(Figure~\ref{fig-m3} a, b). Total insurance was dominated by average
selection (Figure~\ref{fig-m3} c), but spatial insurance was also
positive (Figure~\ref{fig-m3} c). The dominance of average selection was
due to \emph{Sargassum muticum} which had, on average, two to three
times more cover than the other species in monoculture (Appendix 1,
Figure S 1a, b). Thus, even though there was positive covariance between
monoculture functioning and relative abundance for three species
(\emph{Bifurcaria bifurcata}, \emph{Fucus serratus} and \emph{Laminaria
digitata}) across space (the signature of spatial insurance, Appendix 1,
Figure S 1c), \emph{S. muticum} still dominated total insurance. We
found no evidence for temporal or spatio-temporal insurance
(Figure~\ref{fig-m3} c, Appendix 1, Figure S 1d).

\begin{figure}

\centering{

\includegraphics{figures/main_fig_3_proc.png}

}

\caption{\label{fig-m3}Biodiversity effects on cover (\%) for the
rockpool macroalgae (case study 1). \textbf{(a)} net biodiversity
effect, total complementarity (TC) and total selection, \textbf{(b)}
non-random overyielding (NO) and total insurance, and \textbf{(c)}
average selection (AS), spatial insurance (SI), temporal insurance (TI)
and spatio-temporal insurance (ST). Diamonds with error bars represent
the mean and \(95\%\) highest posterior density interval (\(HPDI\)). In
addition, 100 random samples (unfilled circles) are also plotted. The
y-axis scales differ between the three plots, but the red vertical
dashed line indicates a common point at 30\% for a visual reference.}

\end{figure}%

\subsubsection{Case study 2: Marine fouling
communities}\label{case-study-2-marine-fouling-communities-1}

The net biodiversity effect ranged from strong and positive (Mean,
\(HPDI_{95\%}\): 57.5, 50.7 -- 63.3 g) to weak and negative (Mean,
\(HPDI_{95\%}\): -1.03, \(-2.28 - 0.093\) g) across the nine clusters.
This indicates that, depending on the cluster, going from one to five
OTUs could increase biomass by 57.5 g or decrease it by 1.03 g across
times and places, on average. However, pooled across all clusters, the
average net biodiversity effect was significantly positive (Mean,
\(CI_{95\%}\): 24.3, \(10.7 - 38\) g, Appendix 1, Table S 2). As with
the first case study, the net biodiversity effect was dominated by total
complementarity, (Figure~\ref{fig-m4} a) and the mixtures frequently had
higher biomass than any of the monocultures at many times and places
(Appendix 1, Figure S 2).

Total insurance effects were strong and positive and of similar
magnitude to total complementarity (Figure~\ref{fig-m4} a, b). Total
insurance was dominated by average selection (Figure~\ref{fig-m4} c).
This was due to the high monoculture dry biomass and dominance in
mixture of the Barn OTU (\emph{Amphibalanus improvisus}), (Appendix 1,
Figure S 2). Spatial insurance was weaker (Figure~\ref{fig-m4} c) but
still significantly positive across clusters (pooled mean,
\(CI_{95\%}\): 1.7, \(0.60 - 2.9\) g, Appendix 1, Table S 2). We did not
detect temporal nor spatio-temporal insurance effects across clusters
(Appendix 1, Table S 2).

\begin{figure}

\centering{

\includegraphics{figures/main_fig_4_proc.png}

}

\caption{\label{fig-m4}Biodiversity effects on dry biomass (g) for the
marine fouling communities. \textbf{(a)} net biodiversity effect, total
complementarity (TC) and total selection, \textbf{(b)} non-random
overyielding (NO) and total insurance, and \textbf{(c)} average
selection (AS), spatial insurance (SI), temporal insurance (TI) and
spatio-temporal insurance (ST). Diamonds with error bars represent the
pooled mean and 95\% confidence interval for each effect across the nine
clusters estimated using a random-effect meta-analysis model. Asterisks
indicate that the pooled mean is significantly different from zero
(Appendix 1, Table S 2). For each effect and each of the nine clusters,
the \(95\%\) highest posterior density interval (\(HPDI\), coloured
lines) for the effect is shown along with 100 random samples (unfilled
circles) are also plotted. The y-axis scales differ between the three
plots, but the red vertical dashed line indicates a common point at 30 g
for a visual reference.}

\end{figure}%

Whilst spatial insurance was weak, there was variation among clusters
(Figure~\ref{fig-m4} c). We hypothesised that this could be due to
differences in spatial environmental heterogeneity between clusters if
OTUs respond differently to spatial environmental variation. We found a
positive relationship between cluster-level spatial environmental
heterogeneity and the spatial insurance (Figure~\ref{fig-m5} a). The
average slope \(\beta\) between spatial environmental heterogeneity and
the spatial insurance effect was positive across the 100 000 estimates
of the spatial insurance effect (\(\beta_{mean}\), \(HPDI_{95\%}\):
0.56, \(0.06 - 1.3\), Figure~\ref{fig-m5} b). However, this was
influenced by a strong outlier (cluster H) which was the only site that
had substantial presence of the Ciona OTU at some places (Appendix 1,
Figure S 2). This led to high spatial insurance effects as the Barn OTU
(\emph{A. improvisus}) tended to dominate communities in the other
clusters (Appendix 1, Figure S 2). Redoing the analysis without this
outlier strengthened the relationship (\(\beta_{mean}\),
\(HPDI_{95\%}\): 0.93, \(0.35 - 1.6\), Appendix 1, Figure S 3).

\begin{figure}

\centering{

\includegraphics{figures/main_fig_5.png}

}

\caption{\label{fig-m5}Spatial insurance increases with spatial
environmental heterogeneity in the marine fouling communities
\textbf{(a)} The relationship between spatial environmental
heterogeneity (quantified for each of the nine clusters using
multivariate dispersion of 14 environmental variables) and spatial
insurance (mean and 95\% highest density posterior interval,
\(HPDI_{95\%}\)) across the nine clusters. Light grey circles are 100
random samples from the distribution of 100 000 spatial insurance
estimates for each cluster. The red lines are a sample of the 200 simple
linear regression lines from a total of 100 000 regression lines (one
for each of the 100 000 spatial insurance estimates). \textbf{(b)}
Density plot of the distribution of slope estimates from the 100 000
simple linear regressions of multivariate dispersion on spatial
insurance. Red circle with the error bar is the mean and 95\% highest
density posterior interval.}

\end{figure}%

\subsection{Discussion}\label{discussion}

Our analysis of two natural marine ecosystems allowed us to quantify the
complex ways in which biodiversity affects ecosystem functioning across
space and time. In both systems, most of the net biodiversity effect,
which was positive, was due to two effects. First, total
complementarity, which quantifies local-scale species interactions.
Second, average selection, which quantifies the dominance of high
functioning species across space and time (\emph{Sargassum muticum} and
\emph{Amphibalanus improvisus} in case studies 1 and 2, respectively).
This suggests that, in our case studies, processes that allow
local-scale species coexistence as well as the ability of high
functioning species to access a wide geographic area are important for
maintaining ecosystem functioning. In both ecosystems, we found evidence
that variation in species' responses to the environment in space had a
positive effect (i.e.~positive spatial insurance). Moreover, in the
marine fouling system, spatial insurance increased with spatial
environmental heterogeneity. As a result, dispersal processes that allow
for species sorting in space may also be an important determinant of
functioning. Overall, our approach emphasises the importance of taking a
multi-scale perspective in biodiversity-ecosystem functioning research.

The dominant role of total complementarity and average selection is in
line with substantial experimental work showing an important role of
local-scale species interactions and dominance of high functioning
species (Cardinale et al. 2007; Cardinale et al. 2011; Gamfeldt et al.
2015). In comparison, the relative weakness of spatial insurance and the
absence of temporal and spatio-temporal insurance may be explained by
features of our case studies. First, the rockpool data come from two
shores that were chosen because they had similar environmental
conditions (Lars Gamfeldt, personal observation). Moreover, all three
temporal measurements were taken in the same year, which is short
considering the multidecadal lifespan of the focal algae species (Aberg
1992). The situation is similar in the marine fouling communities. While
we aimed to maximise spatial environmental heterogeneity among the five
sites within a cluster (\emph{Materials and methods}), this
spatio-temporal extent of the clusters was only approximately 0.25 km2,
and the temporal measurements were taken over six weeks. This spatial
extent is lower than previous studies from the region, which found more
species turnover than we found (Berntsson and Jonsson 2003). The
relatively similar environmental conditions across places and times of
both case studies probably provided limited opportunity for spatial and
temporal niche partitioning.

Further evidence that the weak spatial and temporal insurance effects
were due to limited environmental heterogeneity comes from (i) comparing
spatial environmental heterogeneity and spatial insurance across
clusters in the marine fouling communities, and (ii) comparing our
results to the original application of the statistical partition (Isbell
et al. 2018). First, there was a positive relationship between spatial
environmental heterogeneity and spatial insurance in the marine fouling
communities (Fig. 5). This suggests that spatial environmental
heterogeneity increased spatial niche partitioning and, therefore,
spatial insurance. Moreover, when removing cluster H which had
relatively unique environmental conditions compared to the other
clusters, the effect strengthened (Appendix 1, Figure S 3). Second, in
the original application of the statistical partition (Isbell et al.
2018), the authors found weak spatial insurance but strong temporal
insurance. However, spatial environmental heterogeneity in their
grassland experiment had only two levels (+/- nitrogen treatment), but
their data spanned 18 years with a range of different weather conditions
during that time (e.g.~droughts, anomalously wet years). Taken together,
this indicates that the magnitude of insurance effects strongly depends
on the spatial and temporal scales of the data, the environmental
variation that the data span (Gamfeldt et al. 2023; Loreau et al. 2021;
Loreau, Mouquet, and Gonzalez 2003).

Our study indicates that insurance effects should continue to strengthen
as we increase scale and, therefore, environmental heterogeneity in
space and time. Broadly, this means that the biodiversity-ecosystem
function field may have failed to quantify much of the positive effect
of biodiversity on ecosystem functioning because most research has taken
place on small scales with limited environmental heterogeneity (Qiu and
Cardinale 2020; Gonzalez et al. 2020). Identifying the appropriate scale
to evaluate insurance effects, however, requires insight into the focal
study system. If the scale is too small, there will not be enough
environmental heterogeneity to allow species to sort in time or space
(Hart, Usinowicz, and Levine 2017). This was likely the case in our
rockpool experiment. On the other hand, very large scales with high
environmental heterogeneity and dispersal limitation will often be
associated with high turnover in the dominance of species. At these
scales, concluding that insurance effects are important is, in our
opinion, trivial. Moreover, it will be difficult, or even untenable, to
quantify biodiversity effects because the biodiversity partition relies
heavily on monoculture data (which will then not be available for many
places and times). Quantifying insurance effects thus requires a match
between the scale of environmental heterogeneity and the way that
species respond to that heterogeneity.

In both case studies, total complementarity was strong (Figures 3 and
4). Thus, species in mixtures, on average, had higher functioning than
expected based on their functioning in monoculture. In particular,
functioning in mixture was higher than even the highest functioning
monocultures at many times and places (Appendix 1, Figures S 1 and S 2).
There are at least three non-mutually exclusive causes for this. First,
our case studies took place in natural environments which may have
higher local-scale environmental heterogeneity than many experimental
systems that often occur in highly controlled conditions. Second,
monocultures were created by removing species whilst mixtures were
undisturbed. This could have inflated the mixture functioning relative
to the monoculture functioning. We did not directly include controls for
biomass removal (Stachowicz et al. 2008; Bracken et al. 2017) but
mitigated this problem by initially removing all algal biomass in the
intertidal rockpool macroalgae study and by scaling monoculture biomass
by area covered in the fouling community study (Appendix 4.4, Figure S
13). Finally, there may simply be strong niche partitioning between
species in our two systems. Indeed, in an experiment with fouling
communities on the east coast of the USA, over time, species-rich
communities occupied the available space on the experimental panels to a
considerably higher degree compared to communities with only individual
species (Stachowicz, Whitlatch, and Osman 1999). Species-rich panels had
higher cover because different species peaked in cover during different
time periods of the experiment (Stachowicz et al. 2008). Our data,
however, do not allow us to mechanistically explain the strong total
complementarity that we observed.

The approach we used to incorporate uncertainty in relative expected
yields worked well on simulated data (Appendix 3, Figure S 11) and, in
case study 2, we were able to obtain relatively accurate monoculture
imputations (Appendix 4.6, Figure S 17 and S 18). We are thus confident
that we were able to estimate all biodiversity insurance effects, and
their relative magnitude, with high accuracy for both case studies. Our
methods provide a promising methodological pipeline for future studies,
allowing researchers to work with incomplete data from natural systems
to estimate biodiversity effects at multiple scales of space and time.

The most challenging part of applying our approach to incomplete data
from natural systems is obtaining sufficient monoculture data. Some
ecological systems are not as amenable to manipulation as our systems
were, which means it may not be possible to even obtain patchy
monoculture data. One option to estimate monoculture functioning when no
direct monoculture data are available is to use more advanced
statistical approaches such as the Joint Species Distribution Modelling
framework (Ovaskainen et al. 2017) or the Diversity-Interactions
Modelling framework (Connolly et al. 2013). These methods simultaneously
model all species and their interactions and as a result can, in
principle, generate monoculture functioning predictions based purely on
mixture data (as suggested by Isbell et al. 2018). The problem is that
statistical approaches based only on mixture data will inevitably
predict beyond the range of the data. However, with mixture data that
covers a wide range of species compositions (e.g.~many two-species
mixtures), it may be feasible. Another option is to field-parameterise
population models as is common in studies of species coexistence (Levine
and HilleRisLambers 2009). If they are set up so that modelled
abundances correspond to or can be easily converted to the ecosystem
function of interest, these models can be used to generate monoculture
predictions by setting interspecific competition to zero in the
parameterised model. Generally, these types of models are difficult to
parameterise, but recent advances using functional traits and species
abundance data may facilitate this in the future (Chalmandrier et al.
2021, 2022). These (or other) statistical or theory-driven approaches
may be the only options if researchers want to calculate these
biodiversity effects in systems where monocultures cannot be
artificially created.

\subsubsection{Conclusions}\label{conclusions}

By developing a methodological pipeline built around a rigorously
derived statistical partition (Isbell et al. 2018), we were able to
dissect the various ways in which biodiversity can affect ecosystem
functioning in two natural marine ecosystems at large spatial and
temporal. This not only showed that biodiversity effects at large scales
were strong and positive but also provided insights into the mechanisms
responsible for these positive effects, namely: local-scale species
interactions, local-scale dominance by high functioning species, and
spatial niche partitioning all contributed to the positive effect that
biodiversity had on ecosystem functioning. If, like in many previous
studies (Thompson et al. 2018, 2021; Chisholm et al. 2013), we would
have only analysed the slope of the relationship between biodiversity
and functioning, all we could have concluded is that the slope does, or
does not, vary with scale, which yields little insight. Overall, our
results suggest that if we want to fully understand how biodiversity
affects the functioning of ecosystems at a range of spatial and temporal
scales, we must move beyond simply correlating biodiversity with
measures of ecosystem functioning. Our pipeline provides a way of doing
that.

\subsection{Acknowledgements}\label{acknowledgements}

We would like to thank Per Jonsson and Jon Havenhand for help with the
experimental design and for reviewing initial drafts of the manuscript.
Per Bergström is thanked for providing access to the GIS data. We would
like to thank Mael Grosse for help with species identification and
photography for case study 2. Finally, a special thanks to John Griffin,
Adam Bates, Matthew McHugh, Patricia Masterson, and volunteers at the
Marine Biological Association of the UK in Plymouth for help with case
study 1, and Julia Murata, Elena Schrofner-Brunner, Lara Martins, Pedro
Victorino de Almeida Dubois de La Patellière and Jesper Hassellöv for
help collecting the data for case study 2. The project was supported by
a doctoral studentship from the University of Gothenburg awarded to
James Hagan and a research grant 2020-03521 from the Swedish Research
Council VR awarded to Lars Gamfeldt.

\subsection{Author contributions}\label{author-contributions}

JH conceived the idea with input from BSB and LG. JH, BSB and LG
designed the experiments. LG led the data collection for case study 1,
and BSB led the data collection for case study 2. JH analysed the data
with input from BSB. JH wrote the first draft with input from BSB and
LG. All authors helped revise the manuscript.

\subsection{Conflict of interest
statement}\label{conflict-of-interest-statement}

We declare no conflicts of interest.

\subsection{Reference list}\label{reference-list}

\phantomsection\label{refs}
\begin{CSLReferences}{1}{0}
\bibitem[\citeproctext]{ref-aberg1992}
Aberg, Per. 1992. {``Size-Based Demography of the Seaweed Ascophyllum
Nodosum in Stochastic Environments.''} \emph{Ecology} 73 (4):
1488--1501. \url{https://doi.org/10.2307/1940692}.

\bibitem[\citeproctext]{ref-barry2021}
Barry, Kathryn E., Gabriella A. Pinter, Joseph W. Strini, Karrisa Yang,
Istvan G. Lauko, Stefan A. Schnitzer, Adam T. Clark, et al. 2021. {``A
Graphical Null Model for Scaling Biodiversity{\textendash}ecosystem
Functioning Relationships.''} \emph{Journal of Ecology} 109 (3):
1549--60. \url{https://doi.org/10.1111/1365-2745.13578}.

\bibitem[\citeproctext]{ref-berntsson2003}
Berntsson, Kent M, and Per R and Jonsson. 2003. {``Temporal and Spatial
Patterns in Recruitment and Succession of a Temperate Marine Fouling
Assemblage: A Comparison of Static Panels and Boat Hulls During the
Boating Season.''} \emph{Biofouling} 19 (3): 187--95.
\url{https://doi.org/10.1080/08927014.2003.10382981}.

\bibitem[\citeproctext]{ref-bracken2017}
Bracken, Matthew E. S., James G. Douglass, Valerie Perini, and Geoffrey
C. Trussell. 2017. {``Spatial Scale Mediates the Effects of Biodiversity
on Marine Primary Producers.''} \emph{Ecology} 98 (5): 1434--43.
\url{https://doi.org/10.1002/ecy.1812}.

\bibitem[\citeproctext]{ref-cardinale2011}
Cardinale, Bradley J. 2011. {``Biodiversity Improves Water Quality
Through Niche Partitioning.''} \emph{Nature} 472 (7341): 86--89.
\url{https://doi.org/10.1038/nature09904}.

\bibitem[\citeproctext]{ref-cardinale2012}
Cardinale, Bradley J., J. Emmett Duffy, Andrew Gonzalez, David U.
Hooper, Charles Perrings, Patrick Venail, Anita Narwani, et al. 2012.
{``Biodiversity Loss and Its Impact on Humanity.''} \emph{Nature} 486
(7401): 59--67. \url{https://doi.org/10.1038/nature11148}.

\bibitem[\citeproctext]{ref-cardinale2011a}
Cardinale, Bradley J., Kristin L. Matulich, David U. Hooper, Jarrett E.
Byrnes, Emmett Duffy, Lars Gamfeldt, Patricia Balvanera, Mary I.
O'Connor, and Andrew Gonzalez. 2011. {``The Functional Role of Producer
Diversity in Ecosystems.''} \emph{American Journal of Botany} 98 (3):
572--92. \url{https://doi.org/10.3732/ajb.1000364}.

\bibitem[\citeproctext]{ref-cardinale2007}
Cardinale, Bradley J., Justin P. Wright, Marc W. Cadotte, Ian T.
Carroll, Andy Hector, Diane S. Srivastava, Michel Loreau, and Jerome J.
Weis. 2007. {``Impacts of Plant Diversity on Biomass Production Increase
Through Time Because of Species Complementarity.''} \emph{Proceedings of
the National Academy of Sciences} 104 (46): 18123--28.
\url{https://doi.org/10.1073/pnas.0709069104}.

\bibitem[\citeproctext]{ref-castillioni2023}
Castillioni, Karen, and Forest Isbell. 2023. {``Early Positive Spatial
Selection Effects of Beta-Diversity on Ecosystem Functioning.''}
\emph{Landscape Ecology} 38 (12): 4483--97.
\url{https://doi.org/10.1007/s10980-023-01786-9}.

\bibitem[\citeproctext]{ref-chalmandrier2021}
Chalmandrier, Loïc, Florian Hartig, Daniel C. Laughlin, Heike Lischke,
Maximilian Pichler, Daniel B. Stouffer, and Loïc Pellissier. 2021.
{``Linking Functional Traits and Demography to Model Species-Rich
Communities.''} \emph{Nature Communications} 12 (1): 2724.
\url{https://doi.org/10.1038/s41467-021-22630-1}.

\bibitem[\citeproctext]{ref-chalmandrier2022}
Chalmandrier, Loïc, Daniel B. Stouffer, Adam S. T. Purcell, William G.
Lee, Andrew J. Tanentzap, and Daniel C. Laughlin. 2022. {``Predictions
of Biodiversity Are Improved by Integrating Trait-Based Competition with
Abiotic Filtering.''} \emph{Ecology Letters} 25 (5): 1277--89.
\url{https://doi.org/10.1111/ele.13980}.

\bibitem[\citeproctext]{ref-chase2013}
Chase, Jonathan M., and Tiffany M. Knight. 2013. {``Scale-Dependent
Effect Sizes of Ecological Drivers on Biodiversity: Why Standardised
Sampling Is Not Enough.''} \emph{Ecology Letters} 16 (s1): 17--26.
\url{https://doi.org/10.1111/ele.12112}.

\bibitem[\citeproctext]{ref-chisholm2013}
Chisholm, Ryan A., Helene C. Muller-Landau, Kassim Abdul Rahman, Daniel
P. Bebber, Yue Bin, Stephanie A. Bohlman, Norman A. Bourg, et al. 2013.
{``Scale-Dependent Relationships Between Tree Species Richness and
Ecosystem Function in Forests.''} \emph{Journal of Ecology} 101 (5):
1214--24. \url{https://doi.org/10.1111/1365-2745.12132}.

\bibitem[\citeproctext]{ref-connolly2013}
Connolly, John, Thomas Bell, Thomas Bolger, Caroline Brophy, Timothee
Carnus, John A. Finn, Laura Kirwan, et al. 2013. {``An Improved Model to
Predict the Effects of Changing Biodiversity Levels on Ecosystem
Function.''} \emph{Journal of Ecology} 101 (2): 344--55.
\url{https://doi.org/10.1111/1365-2745.12052}.

\bibitem[\citeproctext]{ref-craven2020}
Craven, Dylan, Masha T. van der Sande, Carsten Meyer, Katharina
Gerstner, Joanne M. Bennett, Darren P. Giling, Jes Hines, et al. 2020.
{``A Cross-Scale Assessment of Productivity{\textendash}diversity
Relationships.''} \emph{Global Ecology and Biogeography} 29 (11):
1940--55. \url{https://doi.org/10.1111/geb.13165}.

\bibitem[\citeproctext]{ref-creed2009}
Creed, Robert P., Robert P. Cherry, James R. Pflaum, and Chris J. Wood.
2009. {``Dominant Species Can Produce a Negative Relationship Between
Species Diversity and Ecosystem Function.''} \emph{Oikos} 118 (5):
723--32. \url{https://doi.org/10.1111/j.1600-0706.2008.17212.x}.

\bibitem[\citeproctext]{ref-delaplante2011}
deLaplante, Kevin, and Valentin Picasso. 2011. {``The
Biodiversity-Ecosystem Function Debate in Ecology.''} In, edited by
Kevin deLaplante, Bryson Brown, and Kent A. Peacock, 11:169--200.
Handbook of the Philosophy of Science. Amsterdam: North-Holland.
\url{https://doi.org/10.1016/B978-0-444-51673-2.50007-8}.

\bibitem[\citeproctext]{ref-fox2005}
Fox, Jeremy W. 2005. {``Interpreting the {`}Selection Effect{'} of
Biodiversity on Ecosystem Function.''} \emph{Ecology Letters} 8 (8):
846--56. \url{https://doi.org/10.1111/j.1461-0248.2005.00795.x}.

\bibitem[\citeproctext]{ref-frank2022}
Frank, David M. 2022. {``Science and Values in the
Biodiversity-Ecosystem Function Debate.''} \emph{Biology \& Philosophy}
37 (2): 7. \url{https://doi.org/10.1007/s10539-022-09835-4}.

\bibitem[\citeproctext]{ref-gamfeldt2023}
Gamfeldt, Lars, James G. Hagan, Anne Farewell, Martin Palm, Jonas
Warringer, and Fabian Roger. 2023. {``Scaling-up the
Biodiversity{\textendash}ecosystem Functioning Relationship: The Effect
of Environmental Heterogeneity on Transgressive Overyielding.''}
\emph{Oikos} 2023 (3): e09652. \url{https://doi.org/10.1111/oik.09652}.

\bibitem[\citeproctext]{ref-gamfeldt2015}
Gamfeldt, Lars, Jonathan S. Lefcheck, Jarrett E. K. Byrnes, Bradley J.
Cardinale, J. Emmett Duffy, and John N. Griffin. 2015. {``Marine
Biodiversity and Ecosystem Functioning: What's Known and What's Next?''}
\emph{Oikos} 124 (3): 252--65. \url{https://doi.org/10.1111/oik.01549}.

\bibitem[\citeproctext]{ref-gamfeldt2013}
Gamfeldt, Lars, Tord Snäll, Robert Bagchi, Micael Jonsson, Lena
Gustafsson, Petter Kjellander, María C. Ruiz-Jaen, et al. 2013.
{``Higher Levels of Multiple Ecosystem Services Are Found in Forests
with More Tree Species.''} \emph{Nature Communications} 4 (1): 1340.
\url{https://doi.org/10.1038/ncomms2328}.

\bibitem[\citeproctext]{ref-gonzalez2020}
Gonzalez, Andrew, Rachel M. Germain, Diane S. Srivastava, Elise Filotas,
Laura E. Dee, Dominique Gravel, Patrick L. Thompson, et al. 2020.
{``Scaling-up Biodiversity-Ecosystem Functioning Research.''}
\emph{Ecology Letters} 23 (4): 757--76.
\url{https://doi.org/10.1111/ele.13456}.

\bibitem[\citeproctext]{ref-greeve2023}
Greeve, Youk, Per Bergström, Åsa Strand, and Mats Lindegarth. 2023.
{``Estimating and Scaling-up Biomass and Abundance of Epi- and Infaunal
Bivalves in a Swedish Archipelago Region: Implications for Ecological
Functions and Ecosystem Services.''} \emph{Frontiers in Marine Science}
10 (January). \url{https://doi.org/10.3389/fmars.2023.1105999}.

\bibitem[\citeproctext]{ref-hagan2024}
Hagan, James G. 2024. {``Compensation Alters Estimates of the Number of
Species Required to Maintain Ecosystem Functioning Across an Emersion
Gradient: A Case Study with Intertidal Macroalgae.''} \emph{Functional
Ecology} 38 (2): 338--49. \url{https://doi.org/10.1111/1365-2435.14487}.

\bibitem[\citeproctext]{ref-hart2017}
Hart, Simon P., Jacob Usinowicz, and Jonathan M. Levine. 2017. {``The
Spatial Scales of Species Coexistence.''} \emph{Nature Ecology \&
Evolution} 1 (8): 1066--73.
\url{https://doi.org/10.1038/s41559-017-0230-7}.

\bibitem[\citeproctext]{ref-hong2022}
Hong, Pubin, Bernhard Schmid, Frederik De Laender, Nico Eisenhauer,
Xingwen Zhang, Haozhen Chen, Dylan Craven, et al. 2022. {``Biodiversity
Promotes Ecosystem Functioning Despite Environmental Change.''}
\emph{Ecology Letters} 25 (2): 555--69.
\url{https://doi.org/10.1111/ele.13936}.

\bibitem[\citeproctext]{ref-isbell2018}
Isbell, Forest, Jane Cowles, Laura E. Dee, Michel Loreau, Peter B.
Reich, Andrew Gonzalez, Andy Hector, and Bernhard Schmid. 2018.
{``Quantifying Effects of Biodiversity on Ecosystem Functioning Across
Times and Places.''} \emph{Ecology Letters} 21 (6): 763--78.
\url{https://doi.org/10.1111/ele.12928}.

\bibitem[\citeproctext]{ref-isbell2017}
Isbell, Forest, Andrew Gonzalez, Michel Loreau, Jane Cowles, Sandra
Díaz, Andy Hector, Georgina M. Mace, et al. 2017. {``Linking the
Influence and Dependence of People on Biodiversity Across Scales.''}
\emph{Nature} 546 (7656): 65--72.
\url{https://doi.org/10.1038/nature22899}.

\bibitem[\citeproctext]{ref-jochum2020}
Jochum, Malte, Markus Fischer, Forest Isbell, Christiane Roscher, Fons
van der Plas, Steffen Boch, Gerhard Boenisch, et al. 2020. {``The
Results of Biodiversity{\textendash}ecosystem Functioning Experiments
Are Realistic.''} \emph{Nature Ecology \& Evolution} 4 (11): 1485--94.
\url{https://doi.org/10.1038/s41559-020-1280-9}.

\bibitem[\citeproctext]{ref-ggpubr}
Kassambara, Alboukadel. 2023. {``Ggpubr: 'Ggplot2' Based Publication
Ready Plots.''} \url{https://CRAN.R-project.org/package=ggpubr}.

\bibitem[\citeproctext]{ref-koricheva2013}
Koricheva, Julia, Jessica Gurevitch, and Kerrie Mengersen. 2013.
\emph{Handbook of Meta-Analysis in Ecology and Evolution}. Princeton
University Press.

\bibitem[\citeproctext]{ref-leprovost2023}
Le Provost, Gaëtane, Noëlle V. Schenk, Caterina Penone, Jan Thiele,
Catrin Westphal, Eric Allan, Manfred Ayasse, et al. 2023. {``The Supply
of Multiple Ecosystem Services Requires Biodiversity Across Spatial
Scales.''} \emph{Nature Ecology \& Evolution} 7 (2): 236--49.
\url{https://doi.org/10.1038/s41559-022-01918-5}.

\bibitem[\citeproctext]{ref-levine2009}
Levine, Jonathan M., and Janneke HilleRisLambers. 2009. {``The
Importance of Niches for the Maintenance of Species Diversity.''}
\emph{Nature} 461 (7261): 254--57.
\url{https://doi.org/10.1038/nature08251}.

\bibitem[\citeproctext]{ref-lisner2023}
Lisner, Aleš, Marie Konečná, Petr Blažek, and Jan Lepš. 2023.
{``Community Biomass Is Driven by Dominants and Their Characteristics
{\textendash} The Insight from a Field Biodiversity Experiment with
Realistic Species Loss Scenario.''} \emph{Journal of Ecology} 111 (1):
240--50. \url{https://doi.org/10.1111/1365-2745.14029}.

\bibitem[\citeproctext]{ref-loreau2021}
Loreau, Michel, Matthieu Barbier, Elise Filotas, Dominique Gravel,
Forest Isbell, Steve J. Miller, Jose M. Montoya, et al. 2021.
{``Biodiversity as Insurance: From Concept to Measurement and
Application.''} \emph{Biological Reviews} 96 (5): 2333--54.
\url{https://doi.org/10.1111/brv.12756}.

\bibitem[\citeproctext]{ref-loreau2001}
Loreau, Michel, and Andy Hector. 2001. {``Partitioning Selection and
Complementarity in Biodiversity Experiments.''} \emph{Nature} 412
(6842): 72--76. \url{https://doi.org/10.1038/35083573}.

\bibitem[\citeproctext]{ref-loreau2003}
Loreau, Michel, Nicolas Mouquet, and Andrew Gonzalez. 2003.
{``Biodiversity as Spatial Insurance in Heterogeneous Landscapes.''}
\emph{Proceedings of the National Academy of Sciences} 100 (22):
12765--70. \url{https://doi.org/10.1073/pnas.2235465100}.

\bibitem[\citeproctext]{ref-masterson2008}
Masterson, P., F. A. Arenas, R. C. Thompson, and S. R. Jenkins. 2008.
{``Interaction of Top down and Bottom up Factors in Intertidal
Rockpools: Effects on Early Successional Macroalgal Community
Composition, Abundance and Productivity.''} \emph{Journal of
Experimental Marine Biology and Ecology} 363 (1): 12--20.
\url{https://doi.org/10.1016/j.jembe.2008.06.001}.

\bibitem[\citeproctext]{ref-mcgill2015}
McGill, Brian J., Maria Dornelas, Nicholas J. Gotelli, and Anne E.
Magurran. 2015. {``Fifteen Forms of Biodiversity Trend in the
Anthropocene.''} \emph{Trends in Ecology \& Evolution} 30 (2): 104--13.
\url{https://doi.org/10.1016/j.tree.2014.11.006}.

\bibitem[\citeproctext]{ref-vegan}
Oksanen, Jari, Gavin L. Simpson, F. Guillaume Blanchet, Roeland Kindt,
Pierre Legendre, Peter R. Minchin, R. B. O'Hara, et al. 2024. {``Vegan:
Community Ecology Package.''}
\url{https://CRAN.R-project.org/package=vegan}.

\bibitem[\citeproctext]{ref-ovaskainen2017}
Ovaskainen, Otso, Gleb Tikhonov, Anna Norberg, F. Guillaume Blanchet,
Leo Duan, David Dunson, Tomas Roslin, and Nerea Abrego. 2017. {``How to
Make More Out of Community Data? A Conceptual Framework and Its
Implementation as Models and Software.''} \emph{Ecology Letters} 20 (5):
561--76. \url{https://doi.org/10.1111/ele.12757}.

\bibitem[\citeproctext]{ref-petchey2003}
Petchey, Owen L. 2003. {``Integrating Methods That Investigate How
Complementarity Influences Ecosystem Functioning.''} \emph{Oikos} 101
(2): 323--30. \url{https://doi.org/10.1034/j.1600-0706.2003.11828.x}.

\bibitem[\citeproctext]{ref-qiu2020}
Qiu, Jiangxiao, and Bradley J. Cardinale. 2020. {``Scaling up
Biodiversity{\textendash}ecosystem Function Relationships Across Space
and over Time.''} \emph{Ecology} 101 (11): e03166.
\url{https://doi.org/10.1002/ecy.3166}.

\bibitem[\citeproctext]{ref-reich2001}
Reich, Peter B., Jean Knops, David Tilman, Joseph Craine, David
Ellsworth, Mark Tjoelker, Tali Lee, et al. 2001. {``Plant Diversity
Enhances Ecosystem Responses to Elevated CO2 and Nitrogen Deposition.''}
\emph{Nature} 410 (6830): 809--10.
\url{https://doi.org/10.1038/35071062}.

\bibitem[\citeproctext]{ref-rueden2017}
Rueden, Curtis T., Johannes Schindelin, Mark C. Hiner, Barry E. DeZonia,
Alison E. Walter, Ellen T. Arena, and Kevin W. Eliceiri. 2017.
{``ImageJ2: ImageJ for the Next Generation of Scientific Image Data.''}
\emph{BMC Bioinformatics} 18 (1): 529.
\url{https://doi.org/10.1186/s12859-017-1934-z}.

\bibitem[\citeproctext]{ref-sears2007}
Sears, Anna L. W., and Peter Chesson. 2007. {``New Methods for
Quantifying the Spatial Storage Effect: An Illustration with Desert
Annuals.''} \emph{Ecology} 88 (9): 2240--47.
\url{https://doi.org/10.1890/06-0645.1}.

\bibitem[\citeproctext]{ref-smith2003}
Smith, Melinda D., and Alan K. Knapp. 2003. {``Dominant Species Maintain
Ecosystem Function with Non-Random Species Loss.''} \emph{Ecology
Letters} 6 (6): 509--17.
\url{https://doi.org/10.1046/j.1461-0248.2003.00454.x}.

\bibitem[\citeproctext]{ref-smith2020}
Smith, Melinda D., Sally E. Koerner, Alan K. Knapp, Meghan L. Avolio,
Francis A. Chaves, Elsie M. Denton, John Dietrich, et al. 2020. {``Mass
Ratio Effects Underlie Ecosystem Responses to Environmental Change.''}
\emph{Journal of Ecology} 108 (3): 855--64.
\url{https://doi.org/10.1111/1365-2745.13330}.

\bibitem[\citeproctext]{ref-spaak2021}
Spaak, Jurg Werner, and Frederik De Laender. 2021. {``Effects of Pigment
Richness and Size Variation on Coexistence, Richness and Function in
Light-Limited Phytoplankton.''} \emph{Journal of Ecology} 109 (6):
2385--94. \url{https://doi.org/10.1111/1365-2745.13645}.

\bibitem[\citeproctext]{ref-stachowicz2008}
Stachowicz, John J., Rebecca J. Best, Matthew E. S. Bracken, and Michael
H. Graham. 2008. {``Complementarity in Marine Biodiversity
Manipulations: Reconciling Divergent Evidence from Field and Mesocosm
Experiments.''} \emph{Proceedings of the National Academy of Sciences}
105 (48): 18842--47. \url{https://doi.org/10.1073/pnas.0806425105}.

\bibitem[\citeproctext]{ref-stachowicz1999}
Stachowicz, John J., Robert B. Whitlatch, and Richard W. Osman. 1999.
{``Species Diversity and Invasion Resistance in a Marine Ecosystem.''}
\emph{Science} 286 (5444): 1577--79.
\url{https://doi.org/10.1126/science.286.5444.1577}.

\bibitem[\citeproctext]{ref-rstan}
Stan Development Team. 2024. {``{\textbraceleft}RStan{\textbraceright}:
The {\textbraceleft}r{\textbraceright} Interface to
{\textbraceleft}Stan{\textbraceright}.''} \url{https://mc-stan.org/}.

\bibitem[\citeproctext]{ref-thompson2018}
Thompson, Patrick L., Forest Isbell, Michel Loreau, Mary I. O'Connor,
and Andrew Gonzalez. 2018. {``The Strength of the
Biodiversity{\textendash}ecosystem Function Relationship Depends on
Spatial Scale.''} \emph{Proceedings of the Royal Society B: Biological
Sciences} 285 (1880): 20180038.
\url{https://doi.org/10.1098/rspb.2018.0038}.

\bibitem[\citeproctext]{ref-thompson2021}
Thompson, Patrick L., Sonia Kéfi, Yuval R. Zelnik, Laura E. Dee,
Shaopeng Wang, Claire de Mazancourt, Michel Loreau, and Andrew Gonzalez.
2021. {``Scaling up Biodiversity{\textendash}ecosystem Functioning
Relationships: The Role of Environmental Heterogeneity in Space and
Time.''} \emph{Proceedings of the Royal Society B: Biological Sciences}
288 (1946): 20202779. \url{https://doi.org/10.1098/rspb.2020.2779}.

\bibitem[\citeproctext]{ref-tilman2014}
Tilman, David, Forest Isbell, and Jane M. Cowles. 2014. {``Biodiversity
and Ecosystem Functioning.''} \emph{Annual Review of Ecology, Evolution,
and Systematics} 45 (Volume 45, 2014): 471--93.
\url{https://doi.org/10.1146/annurev-ecolsys-120213-091917}.

\bibitem[\citeproctext]{ref-renv}
Ushey, Kevin, and Hadley Wickham. 2024. {``Renv: Project
Environments.''} \url{https://CRAN.R-project.org/package=renv}.

\bibitem[\citeproctext]{ref-loo}
Vehtari, Aki, Jonah Gabry, Måns Magnusson, Yuling Yao, Paul-Christian
Bürkner, Topi Paananen, and Andrew Gelman. 2024. {``Loo: Efficient
Leave-One-Out Cross-Validation and WAIC for Bayesian Models.''}
\url{https://mc-stan.org/loo/}.

\bibitem[\citeproctext]{ref-metafor}
Viechtbauer, Wolfgang. 2010. {``Conducting Meta-Analyses in
{\textbraceleft}r{\textbraceright} with the
{\textbraceleft}Metafor{\textbraceright} Package''} 36.
\url{https://doi.org/10.18637/jss.v036.i03}.

\bibitem[\citeproctext]{ref-gtools}
Warnes, Gregory R., Ben Bolker, Thomas Lumley, Arni Magnusson, Bill
Venables, Genei Ryodan, and Steffen Moeller. 2023. {``Gtools: Various r
Programming Tools.''} \url{https://CRAN.R-project.org/package=gtools}.

\bibitem[\citeproctext]{ref-ggplot2}
Wickham, Hadley. 2016. {``Ggplot2: Elegant Graphics for Data
Analysis.''} \url{https://ggplot2.tidyverse.org}.

\bibitem[\citeproctext]{ref-dplyr}
Wickham, Hadley, Romain François, Lionel Henry, Kirill Müller, and Davis
Vaughan. 2023. {``Dplyr: A Grammar of Data Manipulation.''}
\url{https://CRAN.R-project.org/package=dplyr}.

\bibitem[\citeproctext]{ref-readr}
Wickham, Hadley, Jim Hester, and Jennifer Bryan. 2024. {``Readr: Read
Rectangular Text Data.''}
\url{https://CRAN.R-project.org/package=readr}.

\bibitem[\citeproctext]{ref-tidyr}
Wickham, Hadley, Davis Vaughan, and Maximilian Girlich. 2024. {``Tidyr:
Tidy Messy Data.''} \url{https://CRAN.R-project.org/package=tidyr}.

\bibitem[\citeproctext]{ref-cowplot}
Wilke, Claus O. 2024. {``Cowplot: Streamlined Plot Theme and Plot
Annotations for 'Ggplot2'.''}
\url{https://CRAN.R-project.org/package=cowplot}.

\bibitem[\citeproctext]{ref-williams2017}
Williams, Laura J., Alain Paquette, Jeannine Cavender-Bares, Christian
Messier, and Peter B. Reich. 2017. {``Spatial Complementarity in Tree
Crowns Explains Overyielding in Species Mixtures.''} \emph{Nature
Ecology \& Evolution} 1 (4): 1--7.
\url{https://doi.org/10.1038/s41559-016-0063}.

\bibitem[\citeproctext]{ref-yachi1999}
Yachi, Shigeo, and Michel Loreau. 1999. {``Biodiversity and Ecosystem
Productivity in a Fluctuating Environment: The Insurance Hypothesis.''}
\emph{Proceedings of the National Academy of Sciences} 96 (4): 1463--68.
\url{https://doi.org/10.1073/pnas.96.4.1463}.

\end{CSLReferences}




\end{document}
