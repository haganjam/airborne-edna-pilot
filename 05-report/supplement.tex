% Options for packages loaded elsewhere
\PassOptionsToPackage{unicode}{hyperref}
\PassOptionsToPackage{hyphens}{url}
\PassOptionsToPackage{dvipsnames,svgnames,x11names}{xcolor}
%
\documentclass[
  letterpaper,
  DIV=11,
  numbers=noendperiod]{scrartcl}

\usepackage{amsmath,amssymb}
\usepackage{iftex}
\ifPDFTeX
  \usepackage[T1]{fontenc}
  \usepackage[utf8]{inputenc}
  \usepackage{textcomp} % provide euro and other symbols
\else % if luatex or xetex
  \usepackage{unicode-math}
  \defaultfontfeatures{Scale=MatchLowercase}
  \defaultfontfeatures[\rmfamily]{Ligatures=TeX,Scale=1}
\fi
\usepackage{lmodern}
\ifPDFTeX\else  
    % xetex/luatex font selection
\fi
% Use upquote if available, for straight quotes in verbatim environments
\IfFileExists{upquote.sty}{\usepackage{upquote}}{}
\IfFileExists{microtype.sty}{% use microtype if available
  \usepackage[]{microtype}
  \UseMicrotypeSet[protrusion]{basicmath} % disable protrusion for tt fonts
}{}
\makeatletter
\@ifundefined{KOMAClassName}{% if non-KOMA class
  \IfFileExists{parskip.sty}{%
    \usepackage{parskip}
  }{% else
    \setlength{\parindent}{0pt}
    \setlength{\parskip}{6pt plus 2pt minus 1pt}}
}{% if KOMA class
  \KOMAoptions{parskip=half}}
\makeatother
\usepackage{xcolor}
\setlength{\emergencystretch}{3em} % prevent overfull lines
\setcounter{secnumdepth}{-\maxdimen} % remove section numbering
% Make \paragraph and \subparagraph free-standing
\makeatletter
\ifx\paragraph\undefined\else
  \let\oldparagraph\paragraph
  \renewcommand{\paragraph}{
    \@ifstar
      \xxxParagraphStar
      \xxxParagraphNoStar
  }
  \newcommand{\xxxParagraphStar}[1]{\oldparagraph*{#1}\mbox{}}
  \newcommand{\xxxParagraphNoStar}[1]{\oldparagraph{#1}\mbox{}}
\fi
\ifx\subparagraph\undefined\else
  \let\oldsubparagraph\subparagraph
  \renewcommand{\subparagraph}{
    \@ifstar
      \xxxSubParagraphStar
      \xxxSubParagraphNoStar
  }
  \newcommand{\xxxSubParagraphStar}[1]{\oldsubparagraph*{#1}\mbox{}}
  \newcommand{\xxxSubParagraphNoStar}[1]{\oldsubparagraph{#1}\mbox{}}
\fi
\makeatother


\providecommand{\tightlist}{%
  \setlength{\itemsep}{0pt}\setlength{\parskip}{0pt}}\usepackage{longtable,booktabs,array}
\usepackage{calc} % for calculating minipage widths
% Correct order of tables after \paragraph or \subparagraph
\usepackage{etoolbox}
\makeatletter
\patchcmd\longtable{\par}{\if@noskipsec\mbox{}\fi\par}{}{}
\makeatother
% Allow footnotes in longtable head/foot
\IfFileExists{footnotehyper.sty}{\usepackage{footnotehyper}}{\usepackage{footnote}}
\makesavenoteenv{longtable}
\usepackage{graphicx}
\makeatletter
\def\maxwidth{\ifdim\Gin@nat@width>\linewidth\linewidth\else\Gin@nat@width\fi}
\def\maxheight{\ifdim\Gin@nat@height>\textheight\textheight\else\Gin@nat@height\fi}
\makeatother
% Scale images if necessary, so that they will not overflow the page
% margins by default, and it is still possible to overwrite the defaults
% using explicit options in \includegraphics[width, height, ...]{}
\setkeys{Gin}{width=\maxwidth,height=\maxheight,keepaspectratio}
% Set default figure placement to htbp
\makeatletter
\def\fps@figure{htbp}
\makeatother
% definitions for citeproc citations
\NewDocumentCommand\citeproctext{}{}
\NewDocumentCommand\citeproc{mm}{%
  \begingroup\def\citeproctext{#2}\cite{#1}\endgroup}
\makeatletter
 % allow citations to break across lines
 \let\@cite@ofmt\@firstofone
 % avoid brackets around text for \cite:
 \def\@biblabel#1{}
 \def\@cite#1#2{{#1\if@tempswa , #2\fi}}
\makeatother
\newlength{\cslhangindent}
\setlength{\cslhangindent}{1.5em}
\newlength{\csllabelwidth}
\setlength{\csllabelwidth}{3em}
\newenvironment{CSLReferences}[2] % #1 hanging-indent, #2 entry-spacing
 {\begin{list}{}{%
  \setlength{\itemindent}{0pt}
  \setlength{\leftmargin}{0pt}
  \setlength{\parsep}{0pt}
  % turn on hanging indent if param 1 is 1
  \ifodd #1
   \setlength{\leftmargin}{\cslhangindent}
   \setlength{\itemindent}{-1\cslhangindent}
  \fi
  % set entry spacing
  \setlength{\itemsep}{#2\baselineskip}}}
 {\end{list}}
\usepackage{calc}
\newcommand{\CSLBlock}[1]{\hfill\break\parbox[t]{\linewidth}{\strut\ignorespaces#1\strut}}
\newcommand{\CSLLeftMargin}[1]{\parbox[t]{\csllabelwidth}{\strut#1\strut}}
\newcommand{\CSLRightInline}[1]{\parbox[t]{\linewidth - \csllabelwidth}{\strut#1\strut}}
\newcommand{\CSLIndent}[1]{\hspace{\cslhangindent}#1}

\usepackage{lineno}
\usepackage{setspace}
\linenumbers
\doublespacing
\setlength\linenumbersep{12pt}
\renewcommand\linenumberfont{\normalfont\tiny\sffamily}
\KOMAoption{captions}{tableheading}
\makeatletter
\@ifpackageloaded{caption}{}{\usepackage{caption}}
\AtBeginDocument{%
\ifdefined\contentsname
  \renewcommand*\contentsname{Table of contents}
\else
  \newcommand\contentsname{Table of contents}
\fi
\ifdefined\listfigurename
  \renewcommand*\listfigurename{List of Figures}
\else
  \newcommand\listfigurename{List of Figures}
\fi
\ifdefined\listtablename
  \renewcommand*\listtablename{List of Tables}
\else
  \newcommand\listtablename{List of Tables}
\fi
\ifdefined\figurename
  \renewcommand*\figurename{Figure

S}
\else
  \newcommand\figurename{Figure

S}
\fi
\ifdefined\tablename
  \renewcommand*\tablename{Table

S}
\else
  \newcommand\tablename{Table

S}
\fi
}
\@ifpackageloaded{float}{}{\usepackage{float}}
\floatstyle{ruled}
\@ifundefined{c@chapter}{\newfloat{codelisting}{h}{lop}}{\newfloat{codelisting}{h}{lop}[chapter]}
\floatname{codelisting}{Listing}
\newcommand*\listoflistings{\listof{codelisting}{List of Listings}}
\captionsetup{labelsep=colon}
\makeatother
\makeatletter
\makeatother
\makeatletter
\@ifpackageloaded{caption}{}{\usepackage{caption}}
\@ifpackageloaded{subcaption}{}{\usepackage{subcaption}}
\makeatother

\ifLuaTeX
  \usepackage{selnolig}  % disable illegal ligatures
\fi
\usepackage{bookmark}

\IfFileExists{xurl.sty}{\usepackage{xurl}}{} % add URL line breaks if available
\urlstyle{same} % disable monospaced font for URLs
\hypersetup{
  colorlinks=true,
  linkcolor={blue},
  filecolor={Maroon},
  citecolor={Blue},
  urlcolor={Blue},
  pdfcreator={LaTeX via pandoc}}


\author{}
\date{}

\begin{document}


\section{Supplementary material}\label{supplementary-material}

\textbf{Manuscript type:} Article

\textbf{Title:} Quantifying how biodiversity affects ecosystem
functioning across space and time in natural marine ecosystems

\textbf{Authors:} James G. Hagan\(^{1, 2, 3}\)*, Benedikt
Schrofner-Brunner\(^{2, 3}\) and Lars Gamfeldt\(^{2, 3, 4}\)

\begin{enumerate}
\def\labelenumi{\arabic{enumi}.}
\tightlist
\item
  Community Ecology Lab, Department of Biology, Vrije Universiteit
  Brussel (VUB), Pleinlaan 2, 1050 Brussels, Belgium\\
\item
  Department of Marine Sciences, University of Gothenburg, Box 461,
  SE-40530, Gothenburg, Sweden\\
\item
  Gothenburg Global Biodiversity Centre, Box 461, SE-40530, Gothenburg,
  Sweden\\
\item
  Centre for Sea and Society, Box 260, SE-40530, Gothenburg, Sweden
\end{enumerate}

\textbf{Corresponding author:} James G. Hagan (james\_hagan@outlook.com)

\textbf{Data availability:} Raw data for case study 1 are available on
\href{https://github.com/haganjam/BEF_quant_scale}{Github} and will be
archived upon publication. Raw data for case study 2 are archived on
\href{https://researchbox.org/843&PEER_REVIEW_passcode=GLGJFF}{ResearchBox}
and will be public upon publication.

\textbf{Code availability:} All code used to perform the analysis can be
found on \href{https://github.com/haganjam/BEF_quant_scale}{Github} and
will be archived on Zenodo upon publication.

\textbf{Key words:} biodiversity, ecosystem function, scale,
semi-natural ecosystem, statistical partition

\subsection{Appendix 1: Supplementary
results}\label{appendix-1-supplementary-results}

\begin{figure}

\centering{

\includegraphics{figures/app_1_fig_s1.png}

}

\caption{\label{fig-s1}The change in \% cover of mixtures (blue stars)
and monocultures of the four macroalgal species (filled circles) over
time at the two shores: \textbf{(a)} Challaborough and \textbf{(b)}
Kingsand. Solid lines connect the data points. The relationship between
\% cover in monoculture and relative abundance in the mixture across
\textbf{(c)} places and \textbf{(d)} times for the four study species.
Solid lines in \textbf{(c)} and \textbf{(d)} are linear regression lines
to aid visualisation only.}

\end{figure}%

\begin{figure}

\centering{

\includegraphics{figures/app_1_fig_s2.pdf}

}

\caption{\label{fig-s2}Dry biomass (g) of the five OTUs in monoculture
and in mixture (blue star symbol) across all places (different panels,
number in the top left) and times for all 39 sites in the nine clusters
(A, B, C, D, E, G, H, I and J). Lines connect the points. Points and
error bars represent the mean and 95\% highest posterior density
interval of the imputed monocultures. When no error bar is present, the
observed value is presented.}

\end{figure}%

\begin{figure}

\centering{

\includegraphics{figures/app_1_fig_s3.png}

}

\caption{\label{fig-s3}The relationship between spatial environmental
heterogeneity as quantified using multivariate dispersion of 14
environmental variables (e.g.~wave exposure, mean and average
temperature variability etc.) and the magnitude of the spatial insurance
effect (mean and 95\% highest density posterior interval) across eight
of the nine clusters \textbf{(a)}. For this analysis, we excluded
cluster H because it behaved considerably differently from the other
clusters. Light grey circles are 100 random samples from the
distribution of 100 000 spatial insurance effect estimates for each
cluster. The red lines are a sample of the 200 simple linear regression
lines from a total of 100 000 regression lines (one for each of the 100
000 spatial insurance effect estimates). Density plot of the
distribution of slope estimates from the 100 000 simple linear
regressions of multivariate dispersion on the spatial insurance effect.
Red circle with the error bar is the mean and 95\% highest density
posterior interval \textbf{(b)}.}

\end{figure}%

\begin{figure}

\centering{

\includegraphics{figures/app_1_fig_s4.png}

}

\caption{\label{fig-s4}Multivariate dispersion between clusters
originally classified as heterogeneous and homogeneous based on a priori
geographic data \textbf{(a)}. Clusters originally classified as
heterogeneous had significantly higher multivariate dispersion than
those originally classified as homogeneous (Welch's t-test:
\(t_{5.5} = 3.4\), \(P = 0.017\)). Shapes and error bars indicate the
mean ± SD. The first two principal components, which account for 70\% of
the variation in the data, of the different clusters (colours) that were
originally designated as heterogeneous (circles) or homogeneous
(triangles) based on 14 z-score standardised environmental variables
\textbf{(b)}. Points with white letters are the centroids of each
cluster and represent the cluster identity (main text, Figure 2).}

\end{figure}%

\begin{figure}

\centering{

\includegraphics{figures/app_1_fig_s5.png}

}

\caption{\label{fig-s5}The first two principal components, which account
for 74\% of the variation in the data, of the different places and times
within the different clusters (colours) based on nine environmental
variables that were available for all places and times: panel to the
seabed (m), panel depth (m), the average, coefficient of variation,
maximum and minimum temperature over the course of the experiment (ºC)
and the average, coefficient of variation and maximum light level over
the course of the experiment (lux) \textbf{(a)}. The loadings of the
nine variables are presented for PC1 \textbf{(b)} and PC2 \textbf{(c)}.}

\end{figure}%

\begin{table}

\caption{\label{tbl-s1}Ecological interpretation of the different
biodiversity effects quantified. A visual interpretation of these
different effects can be found in Figure 1 in the main text.}

\centering{

\includegraphics{figures/app_1_table_s1.pdf}

}

\end{table}%

\begin{table}

\caption{\label{tbl-s2}Results of the random effects meta-analysis model
that we used to test whether the different biodiversity effects differed
significantly from zero in the marine fouling community case study.}

\centering{

\includegraphics{figures/app_1_table_s2.pdf}

}

\end{table}%

\begin{table}

\caption{\label{tbl-s3}Comparison of the seven models (Appendix 4.6)
along with an intercept-only null model (Model 8) used to impute the
missing monoculture data in the marine fouling communities. Model
comparison was conducted using leave-one-out cross validation (LOO)
estimated using Pareto Smoothed Importance Sampling (PSIS). In addition
to the LOO estimates and standard errors, we also report the effective
number of parameters (p-LOO) with standard error, the number of
unconstrained parameters (Pars.) and the percentage of data points with
k-diagnostic values for PSIS greater than 0.5. Bold model is the best
model.}

\centering{

\includegraphics{figures/app_1_table_s3.pdf}

}

\end{table}%

\subsection{Appendix 2: Description of the statistical
partition}\label{appendix-2-description-of-the-statistical-partition}

A full description of the statistical partition is beyond the scope of
this article. But, in brief, Isbell and colleagues statistical partition
was developed for a dataset comprising mixtures of competing species
from a single trophic level (i.e.~horizontal communities (Vellend 2010))
in different spatial locations (places) that were measured at different
time points (times) (Isbell et al. 2018). In addition, the dataset had
to contain corresponding monoculture functioning data for all species in
mixture for each place-time combination. With such data we can calculate
a net biodiversity effect (\(NBE\)) for a given mixture as the
difference in the observed functioning of a mixture relative to a null
expectation generated from each species' functioning in monoculture and
their initial relative abundances (i.e.~expected relative yields
\(RY_E\)). For a single mixture with S species at some place and time,
the \(NBE\) (denoted \(NBE_{\alpha}\) to denote that it is calculated
for a single place and time) is (Loreau and Hector 2001):

\[
\mathrm{NBE}_{\alpha} = \sum_{i=1}^{S} \Delta RY_i M_i \tag{Equation S1}
\]

Where \(\Delta RY_{i}\) is the change in relative yield of species \(i\)
(\(RY_{O,i} - RY_{E,i}\)). \(RY_{O,i}\) is the observed relative yield
\(\frac{Y_i}{M_i}\) calculated as the observed functioning in mixture of
species \(i\) (\(Y_i\)) divided by the observed monoculture functioning
of species \(i\) (\(M_i\)), and \(RY_{E, i}\) is the expected relative
yield of species \(i\) in mixture. The \(NBE_{\alpha}\) can then be
partitioned into the local complementary effect (term 1 below) and the
local selection effect (term 2 below):

\[
\mathrm{NBE}_{\alpha} = S \, \overline{\Delta RY} \, \overline{M} + S \, \mathrm{cov}(\Delta RY_i, M_i) \tag{Equation S2}
\]

Isbell and colleagues showed that this equation can be generalised to
calculate an \(NBE\) across \(P\) places and \(T\) times which we denote
\(NBE \gamma\) (following1) as follows (Isbell et al. 2018):

\[
\mathrm{NBE}_{\gamma} = \sum_{k=1}^{P} \sum_{j=1}^{T} \sum_{i=1}^{S} \Delta RY_{ijk} M_{ijk} \tag{Equation S3}
\]

Therefore, \(NBE \gamma\) is the sum of the \(NBE \alpha\) across all
times and places. We can also calculate the total local complementarity
and total local selection effects by simply summing these terms across
all times and places (Isbell et al. 2018). Similarly, this
\(NBE \gamma\) can be partitioned into a total complementarity (term 1)
and a total selection effect (term 2):

\[
\mathrm{NBE}_{\gamma} = PTS \, \overline{\overline{\overline{\Delta RY}}} \, \overline{\overline{\overline{M}}} + PTS \, \mathrm{cov}(\Delta RY_{ijk}, M_{ijk}) \tag{Equation S4}
\]

To quantify the insurance effects of biodiversity, Isbell and colleagues
took this a step further (Isbell et al. 2018). Specifically, they
rewrote the \(NBE \gamma\) not only in terms of \(RY_O\) (relative yield
observed in the mixture) and \(RY_E\) (expected relative yield in
respect of the initial proportions), but also in terms of the observed
relative proportion of species in mixture \(p\). Doing this gives the
following expression for the \(NBE \gamma\) (see1 for details of the
derivation):

\[
\mathrm{NBE}_{\gamma} = PTS \, \overline{\overline{\overline{\Delta RY_{O}}}} \, \overline{\overline{\overline{M}}} + PTS \, \mathrm{cov}(\Delta RY_{O,ijk}, M_{ijk}) + PTS \, \mathrm{cov}(\Delta p_{ijk}, M_{ijk}) \tag{Equation S5}
\]

Where \(\Delta RY_{O, ijk}\) is the difference between the observed
relative yield and observed relative proportion
(\(RY_{O, ijk} - p_{O, ijk}\)). The \(\Delta p_{ijk}\) is the difference
between the observed relative proportion and the expected relative yield
(\(RY_{E, ijk} - p_{O, ijk}\)) and describes the change in dominance in
a mixture of species \(i\) in a given place at a given time. Because
\(\overline{\overline{\overline{\Delta RY_{O}}}} = \overline{\overline{\overline{\Delta RY}}}\),
the first term is equal to the total complementarity effect. The second
term is called non-random overyielding. The third term, which is made up
of the covariance between the change in dominance (\(\Delta p_{ijk}\))
and the monoculture functioning (\(M_{ijk}\)) of species \(i\) in a
given place (\(j\)) at a given time (\(k\)) quantifies the total
insurance effect. The total insurance effect quantifies the extent to
which species that are high functioning in monoculture tend to dominate
mixtures, a common assumption in many models of the insurance effect
(Loreau, Mouquet, and Gonzalez 2003; Yachi and Loreau 1999).

Dominance of species in mixture that are high functioning in monoculture
(the signature of insurance effects) can manifest in several different
ways. Isbell and colleagues showed that the total insurance effect can
be partitioned into four different effects that describe how covariance
between dominance in mixture and high functioning in monoculture
manifests: average selection, spatial insurance, temporal insurance and
spatio-temporal insurance effects (see Figure 1 in the main text and
Table S~\ref{tbl-s1}) (Isbell et al. 2018). For example, the spatial
insurance effect quantifies the part of the \(NBE_{\gamma}\) that is due
to species dominating mixtures in places where they are high functioning
in monoculture. Thus, by quantifying these insurance effects, we can
directly quantify the contribution of species partitioning their niches
in space and time to ecosystem functioning at large spatial and temporal
scales.

\subsection{\texorpdfstring{Appendix 3: Testing the assumption of
unknown expected relative yield values (\(RY_E\)
values)}{Appendix 3: Testing the assumption of unknown expected relative yield values (RY\_E values)}}\label{appendix-3-testing-the-assumption-of-unknown-expected-relative-yield-values-ry_e-values}

As discussed in the previous section, Isbell and colleagues' statistical
partition requires knowing the initial relative abundance of species in
mixture (i.e.~expected relative yields, \(RY_{E}\)), because the
\(RY_{E}\) (along with the functioning of the monocultures) determine
the null expectation of mixture functioning (Isbell et al. 2018).
Specifically, average selection, spatial insurance, temporal insurance
and spatio-temporal insurance are sensitive to the \(RY_{E}\). Thus, by
definition, total insurance, total selection and the net biodiversity
effect are also affected by \(RY_{E}\) because the partition is additive
(see main text Figure 1 and Table S~\ref{tbl-s1}). In contrast, neither
total complementarity nor non-random overyielding are affected by
\(RY_{E}\). In the case of total complementarity, we can show
mathematically that it cannot be affected by only changing \(RY_{E}\).
Consider the case of two species at a single time and single place. The
formula for total complementarity is then:

\[
TC = S \, \overline{\Delta RY} \, \overline{M} \tag{Equation S6}
\]

Where \(S\) is the number of species, \(\overline{\Delta RY}\) is the
average change in relative yield across the two species and
\(\overline{M}\) is the average monoculture functioning across the two
species. The change in relative yield for species, \(\Delta RY_i\), is
calculated as:

\[
\Delta RY_i = RY_{O,i} - RY_{E,i}  
\] \[
RY_{O,i} = \frac{Y_i}{M_i} \tag{Equation S7}
\]

Where \(Y_i\) and \(M_i\) are the functioning values (or yields as it is
commonly referred to in these partitions) of species \(i\) in mixture
and monoculture respectively, \(RY_{O,i}\) is the observed relative
yield of species \(i\) and \(RY_{E,i}\) is the expected relative yield
of species \(i\). Given these equations and this specific case with two
species, we calculate the average change in relative yield
(\(\overline{\Delta RY}\)) across the species as follows:

\[
\overline{\Delta RY} = \frac{(RY_{O,1} - RY_{E,1}) + (RY_{O,2} - RY_{E,2})}{2} \tag{Equation S8}
\]

Doing a simple algebraic rearrangement, we get the following:

\[
\overline{\Delta RY} = \frac{(RY_{O,1} + RY_{O,2}) - (RY_{E,1} + RY_{E,2})}{2} \tag{Equation S9}
\]

Given that the \(RY_E\) values sum to one (Loreau and Hector 2001; Fox
2005), the average change in relative yield (\(\overline{\Delta RY}\))
cannot change by only changing the \(RY_E\). Because calculating total
complementarity is simply the average across all times and places, this
proof holds in the more general case of many species, many times and
many places. In the case of non-random overyielding, it is simply the
case that the \(RY_E\) never enter the calculation (see main text Figure
1 and Table S~\ref{tbl-s1}) and therefore do not affect it.

In classic biodiversity-ecosystem functioning experiments, mixtures are
typically inoculated with equal proportions of all species
(i.e.~mixtures are initially inoculated with a \(\frac{1}{S}\)
proportion of each species where \(S\) is the inoculated number of
species in the mixture). This is referred to as a replacement design
(Loreau, Naeem, and Inchausti 2002). With such an experimental design,
the \(RY_E\) are simply assumed to be \(\frac{1}{S}\) and it is
straightforward to calculate the net biodiversity effect, the selection
effect and the complementarity effect for a given time and place (Loreau
and Hector 2001).

With this partition that considers multiple times and places (Isbell et
al. 2018), choosing the \(RY_E\) is not as straightforward. In the
examples presented by Isbell and colleagues in their Table 2 which have
two species, two places and two times, the \(RY_E\) for the two species
are assumed to be 0.5 at both places and both times (Table
S~\ref{tbl-s4}) (Isbell et al. 2018). The implication of this is not
trivial. It means that even if the observed relative abundances of
species 1 and 2 in place 1 and at time 1 are 0.9 and 0.1 respectively,
the assumed \(RY_E\) for the two species in place 1 and at time 2 would
still be 0.5. This could affect the null expectation of mixture
functioning. Thus, whether to allow the \(RY_E\) to vary through time
and how to do that (e.g.~should it be based on the relative abundances
of species at previous time points?) has not been resolved. Moreover, in
natural systems, there is an additional layer of complexity because we
cannot simply assume that a given place will be colonised by species in
equal proportions as we can with replacement design experiments. Rather,
it is likely that different places will be initially colonised by
species in different proportions. These issues do not, in our opinion,
have a straightforward solution.

Here, we take the following approach to deal with the issues surrounding
the choice of \(RY_E\). First, we assume that all species are initially
present at all times and in all places but that the exact \(RY_E\)
values of each species are unknown. Therefore, we assume a set of random
\(RY_E\) values to use when calculating the different biodiversity
effects (Table S~\ref{tbl-s5}). Second, we do this 100 times and
recalculate the different biodiversity effects with different sets of
\(RY_E\) values. This provides a distribution of possible biodiversity
effects which includes the uncertainty from not knowing the true
\(RY_E\) in each place and at each time. Thus, we follow Isbell and
colleagues in not basing the \(RY_E\) on the relative abundances of
species at previous time points (Isbell et al. 2018). But importantly,
we deviate from Isbell and colleagues by allowing \(RY_E\) to vary and
incorporating that uncertainty into the estimates of the different
biodiversity effects.

\begin{table}

\caption{\label{tbl-s4}The \(RY_E\) assumed by Isbell and colleagues
(2018) in their examples presented in Table 2 of the original paper.}

\centering{

\includegraphics{figures/app_3_table_s4.pdf}

}

\end{table}%

\begin{table}

\caption{\label{tbl-s5}One possible sample of \(RY_E\) using our method
for choosing \(RY_E\) for a dataset with the same structure as Table 2
in the original Isbell and colleagues (2018) paper.}

\centering{

\includegraphics{figures/app_3_table_s5.pdf}

}

\end{table}%

\begin{figure}

\centering{

\includegraphics{figures/app_3_fig_s6.png}

}

\caption{\label{fig-s6}A dataset with the functioning of each of three
species in mixture and in monoculture at two times and two places
\textbf{(a)}. Given unknown \(RY_E\), we draw three \(RY_E\) values from
a Dirichlet distribution and calculate the biodiversity effects that are
sensitive to \(RY_E\), namely: net biodiversity effect, total selection,
total insurance, average selection, spatial insurance, temporal
insurance, spatio-temporal insurance \textbf{(b)}. The result is a
distribution of these effects which reflects the uncertainty in the
biodiversity effects due to not knowing the \(RY_E\) values
\textbf{(c)}.}

\end{figure}%

To do this, we generate a set of random \(RY_E\) values for each place
and time, by drawing a simplex (i.e.~a set of numbers that sum to one)
from a Dirichlet distribution. The Dirichlet distribution is
parameterised using a vector of \(\alpha\)-values that determine how
even the numbers are. In the case of \(RY_E\), the number of
\(\alpha\)-values must be equal to the number of species. Then, using a
set of \(RY_E\) for all places and times (which constitutes one sample,
see example of one sample in Table S~\ref{tbl-s5} for the case of two
species, two places and two times), we can calculate the different
biodiversity effects that are sensitive to the \(RY_E\) (i.e.~net
biodiversity effect, total selection, total insurance, average
selection, spatial insurance, temporal insurance, spatio-temporal
insurance) (Figure S~\ref{fig-s6} for a graphical overview of the
procedure).

We tested if this procedure leads to reliable estimates of the different
biodiversity effects in two different ways. First, we calculated the
different biodiversity effects from the six examples in Table 2 from
Isbell and colleagues assuming that all species were initially present
but with a range of different \(RY_E\) at different places and times
drawn from a Dirichlet distribution (Isbell et al. 2018). These examples
have two species, two times and two places and the \(RY_E\) values were
set at 0.5 for both species. We drew 100 samples for each place-time
combination from a Dirichlet distribution with both \(\alpha\)-values
(one \(\alpha\)-value for each species) equal to three. Preliminary
analyses showed that \(\alpha\)-values of three provide a wide
distribution of possible \(RY_E\) values (Figure S~\ref{fig-s7}). The
variation in \(RY_E\) across the three species decreases with increasing
\(\alpha\), so if we would have reason to assume little variation
between species in their initial relative abundances, we could have
chosen a higher \(\alpha\)-value. We then calculated the biodiversity
effects using these 100 samples and compared the distribution of 100
estimates of each biodiversity effect to the biodiversity effects
presented in Table 2 by Isbell and colleagues (Isbell et al. 2018). In
these six examples, the 95\% percentile interval of the distribution of
the biodiversity effects that are sensitive to \(RY_E\) (net
biodiversity effect, total insurance effect, average selection, spatial
insurance, temporal insurance and spatio-temporal insurance) contained
the true biodiversity effect reported by Isbell and colleagues 100\% of
the time. Moreover, the absolute deviation of the mean of the
distribution of each biodiversity effect from the true biodiversity
effect was a maximum of 2.1 units. This is very low considering the net
biodiversity effect was 100 in their six examples. Therefore, in the
simple examples where the true biodiversity effect is known (Table 2 in
(Isbell et al. 2018)), our method of using random \(RY_E\) was able to
provide an accurate estimate of the different biodiversity effects.

\begin{figure}

\centering{

\includegraphics{figures/app_3_fig_s7.png}

}

\caption{\label{fig-s7}Spread of 200 samples of initial relative
abundances (i.e.~relative expected yields, \(RY_E\)) for three species
drawn from a Dirichlet distribution when assuming alpha values of 1, 2,
3, 4, 5 and 6 (different panels). We chose \(\alpha\)-values of 3 to
parameterise the distribution of \(RY_E\) for our calculations (red
lines) as this provided, in our view, a reasonable spread (Range: 0.02 -
0.90).}

\end{figure}%

However, these examples are very simple. To see if our method of
assuming random \(RY_E\) from the Dirichlet distribution works with more
complex data, we used simulations. For this, we simulated
metacommunities composed of populations of competing species using
Thompson and colleagues' framework implemented in the corresponding R
package mcomsimr (Thompson et al. 2020). We used the population size
(i.e.~the number of individuals, \(N\)) as the measure of functioning as
per the original paper (Thompson et al. 2020). This model incorporates
density-independent species responses to the abiotic environment, inter
and intra-specific competition (we do not include facilitation) and
dispersal. A full explanation of the model assumptions can be found in
Thompson and colleagues (Thompson et al. 2020). But, in brief, the model
considers \(P\) spatially explicit patches inhabited by \(S\)
interacting species. The abiotic environment varies in space and time
and is defined by one variable that varies between 0 and 1, \(E\). The
per capita growth rate of species \(i\) in patch \(p\), at time \(t\)
(\(r_{ip}(t)\)) varies with the environment via the following Gaussian
function:

\[
r_{ip}(t) = r_{\text{max}} \, e^{\left(\frac{z_i - E_p(t)}{2\sigma_i}\right)^2} \tag{Equation S10}
\]

Where \(r_{max}\) is the maximum per capita growth rate of the species.
How close the \(r_{ip}(t)\) gets to the \(r_{max}\) is determined by
\(z_i\) (range 0-1) which is species \(i\)'s abiotic optimum (i.e.~where
its growth rate is maximised), \(\sigma_i\) (range 0-1) which is the
abiotic niche breadth and \(E_p(t)\), (range 0-1) which is the value of
the environment in patch \(p\), at time \(t\). Species abundances within
a patch at a given time (\(N_{ip}(t)\)) are modelled with these growth
rates (\(r_{ip}(t)\)) as inputs into the following logistic population
growth model where species compete according to Lotka-Volterra dynamics
(specifically, the Beverton-Holt formulation):

\[
\dot{N}_{ip}(t + 1) = \max\left(N_{ip}(t) \, \frac{r_{ip}(t)}{1 + \sum_{j=1}^{S} \alpha_{ij} N_{jp}(t)}, \, 0\right) \tag{Equation S11}
\]

In this model, \(N_{ip}(t)\), the population size of species \(i\), in
patch \(p\), at time \(t\) is determined by a combination of the growth
rate (\(r_{ip}(t)\) along with intra- and inter-specific competition
which is modelled using per capita competition coefficients
(\(\alpha_{ij}\)). Finally, dispersal is added to the model via two
additional terms:

\[
N_{ip}(t + 1) = \dot{N}_{ip}(t + 1) - E_{mip}(t) + I_{ip}(t) \tag{Equation S12}
\]

Where \(Em_{ip}(t)\) is the individuals of species \(i\) in patch \(p\)
that leave the patch. \(Em_{ip}(t)\) is determined via the number of
successes in the binomial distribution:
\(\text{Binomial}(N_{ip}(t), a)\), where a is the probability of
success. Thus, the number of dispersing individuals is a stochastic
process. The \(I_{ip}(t)\) term describes the number of individuals of
species \(i\) arriving in patch \(p\), at time \(t\). These individuals
come from the surrounding patches via the following function where the
probability of dispersal of species \(i\) to patch \(p\) depends on the
geographic distance between patches:

\[
I_{ip}(t) = \frac{\sum_{q \ne p}^{P} E_{iy}(t) e^{-L d_{pq}}}{\sum_{p=1}^{P} E_{ip}(t)} \tag{Equation S13}
\]

Here, \(d_{pq}\) is the distance between patches \(p\) and \(q\) whilst
\(L\) describes the strength by which the dispersal probability
decreases with distance. Individuals of species \(i\) leaving patch
\(p\) at time \(t\) (\(Em_{ip}(t)\)) are then randomly allocated to
other patches based on these probabilities that vary with \(d_{pq}\) and
L which gives us the term in equation three above. Finally, following
the addition of the extinction and immigration terms, all species have a
non-zero probability of going locally extinct which is governed by a
binomial distribution with a probability of success: \(pe\).

Using this model, we simulated 1500 different metacommunities with three
species and five patches (i.e.~places in the Isbell and colleagues
terminology) for 300 time-steps. Samples were taken at three time-points
(100, 200 and 300). These simulations were designed to roughly match up
with the data from the two case studies (case study 1: four species, two
places, three times; case study 2: five species, five places per
cluster, three times). Each landscape was initiated as a row of five
patches separated by 25 units and initialised as a torus to avoid edge
effects (Thompson et al. 2020).

\begin{figure}

\centering{

\includegraphics{figures/app_3_fig_s8.png}

}

\caption{\label{fig-s8}Variation in the abiotic environment of the
simulated metacommunities with \textbf{(a)} high spatial environmental
variation and low temporal environmental variation, \textbf{(b)} low
spatial environmental variation but high temporal environmental
variation and \textbf{(c)} a combination of both spatial and temporal
environmental variation. Plotted is one of 500 samples for each of these
three different kinds of simulated environmental variation.}

\end{figure}%

To simulate a continuously varying abiotic environment between 0 and 1,
we used an exponential covariance model as per the original paper
(Thompson et al. 2020). However, unlike Thompson and colleagues we used
this model to simulate three different types of continuously fluctuating
abiotic environments (500 replicates simulations of each type). First,
we aimed to maximise only spatial environmental variation (Figure
S~\ref{fig-s8} a). To do this, we used the exponential covariance model
to simulate a different fluctuating environment for each of the five
places (\(\mu\) = 0.5, \(\sigma\) = 0.25, autocorrelation parameter =
5). We then standardised the fluctuating environments for each place
between 0-0.2, 0.2-0.4, 0.4-0.6, 0.6-0.8 and 0.8-1 respectively. This
meant that the value of the abiotic environment never overlapped between
the five places and allowed some minor temporal environmental variation
(Figure S~\ref{fig-s8} a).

Second, we aimed to maximise temporal, but not spatial, variation. For
this, we used the exponential covariance model to simulate a single
fluctuating environment (\(\mu\) = 0.5, \(\sigma\) = 0.25,
autocorrelation parameter = 5) which we applied to all five places.
Then, for each place, we translated the abiotic environment variable by
an amount drawn from a normal distribution: Normal(0, 0.02). Across all
places, we standardised the abiotic environment variable between 0 and 1
(Figure S~\ref{fig-s8} b).

Third, we aimed to simulate places that had a combination of spatial and
temporal environmental variation. Again, we used the exponential
covariance model to simulate a fluctuating environment for each place
but we drew the mu parameter from a uniform distribution (\(\mu\) =
Uniform(0,1), \(\sigma\) = 0.25, autocorrelation parameter = 5). We
standardised the abiotic environment between 0 and 1 across all places
(Figure S~\ref{fig-s8} c).

\begin{figure}

\centering{

\includegraphics{figures/app_3_fig_s9.png}

}

\caption{\label{fig-s9}Population size as a measure of functioning of
species 1-3 in mixture in a single 5-patch metacommunity simulated for
300 time points and with three time points plotted (100, 200 and 300).}

\end{figure}%

The rest of the model parameters were chosen as follows. Places were
seeded with three species with starting abundances that summed to 150
individuals and with a minimum starting abundance of 10 individuals. The
probability of extirpation (\(pe\)) was set at 0.00001. The dispersal
rate term (\(a\)) was drawn from a uniform distribution:
\(\text{Uniform}(0.01, 0.1)\). Species' abiotic optima (\(z_i\)) were
evenly spaced between 0.3 and 0.7 and species' abiotic niche breadths
(\(\sigma_i\)) were drawn from a uniform distribution:
\(\text{Uniform}(0.15, 0.3)\). The competition coefficients
(\(\alpha_{ij}\)) were drawn from a uniform distribution:
\(\text{Uniform}(\alpha_{min}, \alpha_{max})\) where
\(\alpha_{min} \sim \text{Uniform}(0, 0.5)\) and
\(\alpha_{max} \sim \text{Uniform}(0.5, 1.5)\) and assumed to be
symmetric (i.e.~\(\alpha_{ij} = \alpha_{ji}\)). Intraspecific
competition (i.e.~\(\alpha_{i = j}\)) was set at one and all competition
coefficients were scaled by multiplying by 0.05 to allow higher
equilibrium abundances (Thompson et al. 2020). Finally, for each
simulation, we simulated identical metacommunities with each species in
monoculture seeded into each patch. Thus, each of the 1500 simulated
metacommunities had complete mixture and monoculture data for all places
and time points (see Figure S~\ref{fig-s9} and Figure S~\ref{fig-s10}
for example simulations).

\begin{figure}

\centering{

\includegraphics{figures/app_3_fig_s10.png}

}

\caption{\label{fig-s10}Population size as a measure of functioning of
species 1-3 in monoculture in a single 5-patch metacommunity simulated
for 300 time points and with three time points plotted (100, 200 and
300).}

\end{figure}%

\begin{table}

\caption{\label{tbl-s6}The mean and range (minimum and maximum) of the
estimated biodiversity effects that are sensitive to varying \(RY_E\)
across the 500 simulations of each of three different types of
abiotically varying environment: spatial environmental variation (Figure
S8a), temporal environmental variation (Figure S8b) and a combination of
spatial and temporal environmental variation (Figure S8c).}

\centering{

\includegraphics{figures/app_3_table_s6.pdf}

}

\end{table}%

Using these simulations, we were able to generate metacommunities with
different levels of intra- and inter-specific competition, dispersal
rates and environmental heterogeneity in space and time which led to a
wide variety of different magnitudes of the different biodiversity
effects (Table S~\ref{tbl-s6}). This was sufficient for our purposes as
we aimed to test our workflow against simulated data with known
biodiversity effects rather than test specific hypotheses regarding how
different parameter combinations may affect the different biodiversity
effects.

We recorded the initial relative abundance of the different species in
each place. Using this information, we defined the true, simulated
biodiversity effects (for the effects that are sensitive to changing
\(RY_E\) values, namely: net biodiversity effect, total insurance
effect, average selection, spatial insurance, temporal insurance and
spatio-temporal insurance) as the biodiversity effect assuming that, for
each patch at all three time points, the \(RY_E\) values were the
initial relative abundance of the different species (consistent with the
replacement series design (Loreau, Naeem, and Inchausti 2002)). We note
that this does not fully resolve the issue of whether the \(RY_E\)
should change through time and rather follows the methodology in the
original paper (Isbell et al. 2018) of assuming the same \(RY_E\) values
for all places through time. Then, as described previously (see also
Figure S~\ref{fig-s6}), we drew 100 samples for each place-time
combination from a Dirichlet distribution with all three
\(\alpha\)-values equal to three. We then calculated the biodiversity
effects using these 100 samples and compared them with 100 samples of
the true, simulated biodiversity effects. Finally, we correlated (i) the
biodiversity effects based on \(RY_E\) values from the Dirichlet
distribution and (ii) the true, simulated biodiversity effects. For this
correlation, we pooled the three different environmental variation
scenarios. We pooled the scenarios to see if our method of assuming
random \(RY_E\) values from a Dirichlet distribution works when we do
not know \emph{a priori} the kind of environmental variation that is
present in any given dataset (i.e.~whether the environmental variation
is mainly spatial, temporal or a combination of both).

For all nine biodiversity effects and pooled across the three scenarios,
the Pearson correlation coefficient between the mean estimated
biodiversity effect and the true, simulated biodiversity effect was
between 0.94 and 1 (Figure S~\ref{fig-s11}). Moreover, across all 1500
simulated metacommunities, the 95\% percentile interval of the estimated
distribution of biodiversity effects contained the true, simulated
biodiversity effect in more than 80\% of the simulations for the net
biodiversity effect, total selection, total insurance, temporal
insurance and spatio-temporal insurance barring average selection and
spatial insurance (Figure S~\ref{fig-s11}). For average selection and
spatial insurance, the 95\% percentile interval contained the true,
simulated biodiversity effect in 60\% and 71\% of the 1500 simulations
respectively. However, despite the lower percentage than the other
effects, the absolute error of the estimated mean average selection and
mean spatial insurance effect from the true simulated biodiversity
effects was low (Median: 10.1 and 9.4 respectively, Table
S~\ref{tbl-s7}) considering that the true simulated effects were
relatively high in comparison (Table S~\ref{tbl-s7}). Finally, despite
these relatively minor precision issues, the method of assuming random
\(RY_E\) values consistently estimated the correct relative magnitude of
the different effects (i.e.~which effects were greater or less than
others within a given simulation). Specifically, the median Spearman
rank correlation between the biodiversity effects within each individual
simulation was 1 (\(PI_{95\%}\): 0.79 - 1).

\begin{figure}

\centering{

\includegraphics{figures/app_3_fig_s11.png}

}

\caption{\label{fig-s11}The relationship between the true, simulated
biodiversity effects and the mean and 95\% percentile interval
(\(PI_{95\%}\)) of the estimated biodiversity effects assuming 100
different samples from the Dirichlet distribution for each of the five
patches and time points in the 1500 simulated metacommunities. Colours
indicate whether the true, simulated biodiversity effect is within the
95\% percentile interval for a given metacommunity simulation. The
black, dashed line is the 1:1 line. Also reported is the Pearson's
correlation coefficient for each biodiversity effect (\(r\)).}

\end{figure}%

\begin{table}

\caption{\label{tbl-s7}The proportion of the 1500 simulations where the
true, simulated biodiversity was within the 95\% percentile interval of
the distribution of biodiversity effects from assuming 100 different
\(RY_E\) values for each place drawn from a Dirichlet distribution. Also
reported is the absolute deviation of the mean of the distribution of
biodiversity effects from the true, simulated biodiversity effect across
the 1500 simulations along with the pooled mean of each effect across
the 1500 simulations for comparison with the error values.}

\centering{

\includegraphics{figures/app_3_table_s7.pdf}

}

\end{table}%

\subsection{Appendix 4: Extended methods for case study
2}\label{appendix-4-extended-methods-for-case-study-2}

\subsubsection{Appendix 4.1: Overview and map of the study area and
experimental
design}\label{appendix-4.1-overview-and-map-of-the-study-area-and-experimental-design}

The marine environment around Tjärnö on the Swedish West Coast (Figure
S~\ref{fig-s12}) is subject to various natural and anthropogenic
influences and hosts a mixture of shallow and deep-water habitats,
including rocky reefs, macroalgae forests, seagrass meadows, and sandy
bottoms characterise this area. The tidal range is about 0.3 m, but due
to wind and changes in atmospheric pressure, the water level can
fluctuate by up to 2 m10. Many marine sessile organisms (e.g.,
barnacles, bryozoans, hydrozoans, ascidians) settle opportunistically on
available surfaces of bare rock, macroalgae, and artificial surfaces
such as boat hulls. Previous studies, especially focused on the
mitigation of the biofouling problem, successfully used PMMA (Polymethyl
methacrylate) panels as settling substrate for fouling species
(Berntsson and Jonsson 2003).

\begin{figure}

\centering{

\includegraphics{figures/app_4_fig_s12.png}

}

\caption{\label{fig-s12}Map of the study area with the initial 50 sites
(points) coloured by cluster. Clusters B, F, H, I and J were classified
as homogenous and were, therefore, at the same depth level within
\textbf{(a)}. Clusters A, C, D, E and G were classified as heterogenous
and had alternating depths within clusters (see Figure S4). At each site
12 PMMA panels were mounted on a wire frame in either 3 or 6 m depth
\textbf{(b)}. At each site, three panels were randomly assigned to be
mixtures at different time points (t1-3). After settlement on the
monoculture (mono) panels, the panel was assigned to a species (S1-5)
and all other species besides the target species were removed.}

\end{figure}%

\subsubsection{Appendix 4.2: Choosing sites and
clusters}\label{appendix-4.2-choosing-sites-and-clusters}

When choosing the sites, we aimed to create five environmentally
heterogeneous and five environmentally homogeneous clusters with five
sites each. We did this using a priori geographical data of three
variables: water depth from nautical charts (1:50 000; Sjöfartsverket
2022), turbidity (depth of visibility (m), 15 m resolution) and relative
wave exposure (unitless, 15 m resolution) (Greeve et al. 2023). We set
the minimum distance between sites within a cluster to 40 m and the
maximum distance to approximately 1000 m. Water depth had to be between
7 m and 20 m. Moreover, the clusters were not allowed to overlap and had
to be relatively equally spaced around the experimental area (Figure
S~\ref{fig-s12}). With these constraints, we randomly sampled clusters
of sites and calculated environmental heterogeneity. Environmental
heterogeneity was based on z-score standardised water depth, turbidity
and wave exposure and calculated as multivariate dispersion based on
Euclidean distance using the betadisper() function from the vegan
package (Oksanen et al. 2024) in R v4.1.214. We chose clusters that
varied in their multivariate dispersion. The sites within the five
clusters with the lowest multivariate dispersion were kept at a constant
depth of 3 m or 6 m and the sites within the five clusters with the
highest multivariate dispersion had alternating depths (e.g.~two sites
with 3 m depth and three sites with 6 m depth).

\subsubsection{Appendix 4.3: In-situ environmental variable
measurements}\label{appendix-4.3-in-situ-environmental-variable-measurements}

During the experiment, we measured salinity using a probe (YSI 30
Conductivity Salinity Temperature) and Secchi depth using a Secchi disk
at each site on two separate occasions. The frame depth and distance
from the frame to the seabed at each site was measured on three
occasions which corresponded to the days when mixtures and monocultures
were collected (days 34, 47 and 60 of the experiment). In addition, we
measured relative water movement approximated by gypsum dissolution
(Porter, Sanford, and Suttles 2000) on two occasions during the
experiment at all sites. For this, we installed a gypsum cylinder (mean
= 57.4 g, sd = 1.9 g) at each site at approximately the same time (± 2
h) for 24 h. Since the gypsum dissolution is sensitive to various
variables (surface area, temperature, salinity, type of water motion),
it cannot be used to directly derive water movement (Porter, Sanford,
and Suttles 2000). However, it is a relative indicator of water
movement. Therefore, we can use the raw value (mass reduction g hour-1)
to compare sites in terms of water movement. These two measurements were
averaged across the two time points for each site.

In addition to these measurements, we installed a light-temperature
logger (Onset UA-002-64 HOBO Pendant® Temperature/Light 64K Data and
Onset MX2202 HOBO Pendant® MX Temperature/Light Data Logger) on each
frame with an hourly logging interval during the time of the experiment.
We used these data to calculate the average, coefficient of variation,
maximum and minimum temperature (ºC) and the average, coefficient of
variation and maximum light level (lux) for three separate intervals:
start of the experiment to the first measurements on day 34 (t1),
between day 34 and day 47 (t2) and between day 47 (t3) and the end of
the experiment on day 60.

\subsubsection{Appendix 4.4: Correcting monocultures for the effects of
removal}\label{appendix-4.4-correcting-monocultures-for-the-effects-of-removal}

On the monoculture panels, the OTUs did not always completely cover the
settling panel (Figure S~\ref{fig-s13} b versus almost full coverage
observed in Figure S~\ref{fig-s13} a). This is because we had to scrape
off other OTUs to make sure we could obtain a monoculture (see main text
\emph{Materials and methods}). To correct for this, we calculated the
dry biomass of a monoculture tile assuming that it covered the total
area of the tile. To do this, we divided monoculture biomass by the
proportional cover of the monoculture on the panel:

\[
\textit{corrected dry biomass (g)} = \frac{\textit{dry biomass (g)}}{\textit{proportion monoculture cover}} \tag{Equation S14}
\]

\begin{figure}

\centering{

\includegraphics{figures/app_4_fig_s13.png}

}

\caption{\label{fig-s13}A mixture panel with almost 100 \% cover
\textbf{(a)}. A monoculture panel with 65\% cover of the Bryo OTU
\textbf{(a)}. The margin of 1 cm around the organisms was removed before
measurement and is ignored for the cover calculation.}

\end{figure}%

\subsubsection{Appendix 4.5: Similarity of replicate tiles at the same
time
point}\label{appendix-4.5-similarity-of-replicate-tiles-at-the-same-time-point}

In the marine fouling community experiment, tiles were destructively
sampled at different time points that were meant to represent the same
community (Figure S~\ref{fig-s14}). For example, we had three tiles for
the mixture at each site (i.e.~each buoy). These tiles were
destructively sampled at different time points and were thus meant to
represent the same mixture (Figure S~\ref{fig-s14}). Therefore, we
assumed that, at all three time points, the three different tiles would
contain similar or identical marine fouling communities (see Figure
S~\ref{fig-s15} for a graphical representation of the problem).

\begin{figure}

\centering{

\includegraphics{figures/app_4_fig_s14.png}

}

\caption{\label{fig-s14}Layout of marine fouling tiles at a single buoy
\textbf{(a)}. Tiles are assigned to be mixtures (mix) or monocultures of
different OTUs (mono s1, mono s2 etc.). Moreover, there are three
separate tiles for the mixture and for each monoculture. These tiles are
destructively sampled for biomass measurements at three different time
points (t1, t2 and t3). Before destructive sampling, photographs are
taken of all tiles from which percentage cover of different OTUs (s1-5)
is extracted. At the given time point, the relevant tiles are
destructively sampled for dry biomass (g) \textbf{(b)}.}

\end{figure}%

We cannot test this assumption with the biomass data obtained from the
destructive samples because we only have one destructive sample for each
time point (Figure S~\ref{fig-s14}). However, we photographed all tiles
at all time points. Thus, we can compare the percentage cover of
different OTUs on different tiles at the same time point. Given that
cover and biomass tend to be strongly correlated, we can compare the
cover of OTUs on different tiles at the same time point. This allows us
to test the assumption that the three different tiles that we sampled at
three different time points contain similar or almost identical fouling
communities (Figure S~\ref{fig-s15}).

\begin{figure}

\centering{

\includegraphics{figures/app_4_fig_s15.png}

}

\caption{\label{fig-s15}We assumed that, for a given tile category
(e.g.~mixture, monoculture s1 etc.), the three replicates that were
destructively sampled at the three different times point (t1-3, Figure
S14) have similar community structure \textbf{(a)}. Therefore, we
assumed that the dry biomass of different species correlates well across
time points. But we only have biomass data for t1 tiles at t1 (Figure
S15). Thus, we tested this assumption using the percentage cover values
as we have cover data for tiles at all time-points (Figure S15)
\textbf{(b)}.}

\end{figure}%

To test whether tiles that were destructively sampled at different time
points had similar community structure at the same time points, we chose
20 buoys randomly and extracted the percentage cover of different OTUs
from photographs using ImageJ (Rueden et al. 2017). This was done for
tiles that were destructively sampled at different time points but where
the photographs were taken on the same day. Thus, we had OTU cover (\%)
data taken at the same time from two different tiles from 20 buoys (40
data points in total).

\begin{figure}

\centering{

\includegraphics{figures/app_4_fig_s16.png}

}

\caption{\label{fig-s16}Relationship between cover on mixture tiles
photographed at the same time point for all tiles and all OTUs
(different colours) \textbf{(a)}. Relationship between total cover of
all OTUs on mixture tiles photographed at the same time point
\textbf{(b)}. The dashed line is the 1:1 line. The rho is Spearman's
correlation coefficient.}

\end{figure}%

Across all OTUs and buoys, there was a strong positive correlation
between percentage cover on the different tiles (Spearman's r = 0.82, S
= 30119, P \textless{} 0.001). Most values were close to the 1:1 line,
but there were several outliers (Figure S~\ref{fig-s16}). In addition,
the total cover of all OTUs (i.e.~sum of cover of all species on a tile)
on the different tiles was also strongly correlated (Spearman's r =
0.80, S = 264, P \textless{} 0.001) and the values were close to the 1:1
line (Figure S~\ref{fig-s16}). These strong, positive correlations
indicate that the community structure on different tile replicates
measured at the same time was similar. In line with this, the absolute
deviation of OTU cover between the different tile replicates across OTUs
was relatively low (mean ± SD: 5.3 ± 11 \%). Given that percentage cover
values are well known to contain considerable error compared to more
direct abundance metrics like biomass (Muukkonen et al. 2006), these
results indicate that tiles that were destructively sampled at different
time points do have similar community structure when measured at the
same time in terms of percentage cover. All correlation analyses
presented were performed using the cor() function in R v4.1.214.

\subsubsection{Appendix 4.6: Multilevel regression models for
monoculture
imputation}\label{appendix-4.6-multilevel-regression-models-for-monoculture-imputation}

A complete mixture-monoculture dataset for this experiment would consist
of monocultures of all five OTUs for the 39 mixtures at all three time
points (i.e.~39 places × 3 time points × 5 species = 585 monoculture
datapoints). We imputed this missing data using a generalised linear
model (GLM) fit in a Bayesian framework. To do this, we first used a
Principal Components Analysis (PCA) to decompose nine environmental
variables into their major axes of variation, namely: panel depth (m),
distance from the panel to the seabed (m), the average, coefficient of
variation, maximum and minimum temperature over the course of the
experiment (ºC) and the average, coefficient of variation and maximum
light level over the course of the experiment (lux). All nine variables
were z-score standardised, and the PCA was fit using the prcomp()
function in R v4.1.214. The first two PC axes (PC1 and PC2) explained
74\% of the variation in these nine environmental variables (Figure
S~\ref{fig-s5}).

We then fit seven multilevel GLMs with monoculture biomass (M) as the
response variable and different combinations of PC1, PC2 and the OTU's
biomass in mixture (Y) as predictor variables along with an intercept
only null model (model 8). All predictor variables were standardized
between 0 and 1. The observed monoculture biomasses were strictly
positive, right skewed and contained zeros (ca. 8\% of the observed
monoculture biomass values were zero). Thus, we modelled monoculture
biomass using a Log-Normal hurdle model. A hurdle model is a mixture
model where the response variable is modelled as zero or non-zero (0-1)
using a Bernoulli process. Then, if the response variable is non-zero,
it is modelled using a different distribution, in this case, the
Log-Normal distribution. This allowed us to explicitly model zeros and
the positive, right-skew in the non-zero data.

The seven models and the null model (Model 8) with priors are presented
below. We used weakly informative priors that we chose using a prior
predictive simulation. All models were written using a non-centered
parameterisation (McElreath 2018). For each model, we estimated the
posterior distribution using Stan's No-U-Turn Sampler Hamiltonian Monte
Carlo algorithm (mc-stan.org/) with four separate chains. We implemented
this in R v4.1.214 using the rstan package (2024). We assessed model
convergence by inspecting trace plots, R-hat values (Gelman-Rubin
statistic) and effective sample sizes.

\paragraph{Model 1}\label{model-1}

\[
\begin{aligned}
w_i &\sim \mathrm{Bernoulli}(p_i) \\
\text{logit}(p_i) &= \theta_{S[i]} + \delta_{1S[i]} Y_i \\
\theta_j &= \bar{\theta} + \nu_{.1} \\
\delta_{1j} &= \bar{\delta} + \nu_{.2} \\
\nu &= \left( \mathrm{diag}(\tau) \times \mathrm{cholesky}(R) \times V \right)^\top \\
V_{k,j} &\sim \mathcal{N}(0, 1) \\
\bar{\theta}, \bar{\delta} &\sim \mathcal{N}(0, 1.5) \\
\tau_\theta, \tau_\delta &\sim \mathrm{Exponential}(3) \\
R_{k,k} &\sim \mathrm{LKJcorr}(2) \\
(M_i \mid w_i = 1) &\sim \mathrm{LogNormal}(u_i, \epsilon) \\
u_i &= \alpha_{S[i]} + \beta_{1S[i]} Y_i + \beta_{2S[i]} \mathrm{PC1}_i \\
\alpha_j &= \bar{\alpha} + z_{.1} \\
\beta_{1j} &= \bar{\beta}_1 + z_{.2} \\
\beta_{2j} &= \bar{\beta}_2 + z_{.3} \\
z &= \left( \mathrm{diag}(\sigma) \times \mathrm{cholesky}(L) \times Z \right)^\top \\
\epsilon &\sim \mathrm{Exponential}(5) \\
Z_{l,j} &\sim \mathcal{N}(0, 1) \\
\bar{\alpha}, \bar{\beta}_1, \bar{\beta}_2 &\sim \mathcal{N}(0, 1.5) \\
\sigma_\alpha, \sigma_{\beta_1}, \sigma_{\beta_2} &\sim \mathrm{Exponential}(3) \\
L_{l,l} &\sim \mathrm{LKJcorr}(2) \\
w_i &=
\begin{cases}
0 & \leftrightarrow (M_i = 0) \\
1 & \leftrightarrow (M_i > 0)
\end{cases}
\end{aligned}
\]

\paragraph{Model 2}\label{model-2}

\[
\begin{aligned}
w_i &\sim \mathrm{Bernoulli}(p_i) \\
\text{logit}(p_i) &= \theta + \delta_1 Y_i \\
\theta, \delta_1 &\sim \mathcal{N}(0, 1.5) \\
(M_i \mid w_i = 1) &\sim \mathrm{LogNormal}(u_i, \epsilon) \\
u_i &= \alpha_{S[i]} + \beta_{1S[i]} Y_i + \beta_{2S[i]} \mathrm{PC1}_i \\
\alpha_j &= \bar{\alpha} + z_{.1} \\
\beta_{1j} &= \bar{\beta}_1 + z_{.2} \\
\beta_{2j} &= \bar{\beta}_2 + z_{.3} \\
z &= \left( \mathrm{diag}(\sigma) \times \mathrm{cholesky}(L) \times Z \right)^\top \\
\epsilon &\sim \mathrm{Exponential}(5) \\
Z_{l,j} &\sim \mathcal{N}(0, 1) \\
\bar{\alpha}, \bar{\beta}_1, \bar{\beta}_2 &\sim \mathcal{N}(0, 1.5) \\
\sigma_\alpha, \sigma_{\beta_1}, \sigma_{\beta_2} &\sim \mathrm{Exponential}(3) \\
L_{l,l} &\sim \mathrm{LKJcorr}(2) \\
w_i &=
\begin{cases}
0 & \leftrightarrow (M_i = 0) \\
1 & \leftrightarrow (M_i > 0)
\end{cases}
\end{aligned}
\]

\paragraph{Model 3}\label{model-3}

\[
\begin{aligned}
w_i &\sim \mathrm{Bernoulli}(p_i) \\
\text{logit}(p_i) &= \theta + \delta_1 Y_i \\
\theta, \delta_1 &\sim \mathcal{N}(0, 1.5) \\
(M_i \mid w_i = 1) &\sim \mathrm{LogNormal}(u_i, \epsilon) \\
u_i &= \alpha_{S[i]} + \beta_{1S[i]} Y_i \\
\alpha_j &= \bar{\alpha} + z_{.1} \\
\beta_{1j} &= \bar{\beta}_1 + z_{.2} \\
z &= \left( \mathrm{diag}(\sigma) \times \mathrm{cholesky}(L) \times Z \right)^\top \\
\epsilon &\sim \mathrm{Exponential}(5) \\
Z_{l,j} &\sim \mathcal{N}(0, 1) \\
\bar{\alpha}, \bar{\beta}_1 &\sim \mathcal{N}(0, 1.5) \\
\sigma_{\alpha}, \sigma_{\beta_1} &\sim \mathrm{Exponential}(3) \\
L_{l,l} &\sim \mathrm{LKJcorr}(2) \\
w_i &=
\begin{cases}
0 & \leftrightarrow (M_i = 0) \\
1 & \leftrightarrow (M_i > 0)
\end{cases}
\end{aligned}
\]

\paragraph{Model 4}\label{model-4}

\[
\begin{aligned}
w_i &\sim \mathrm{Bernoulli}(p_i) \\
\text{logit}(p_i) &= \theta + \delta_1 Y_i \\
\theta, \delta_1 &\sim \mathcal{N}(0, 1.5) \\
(M_i \mid w_i = 1) &\sim \mathrm{LogNormal}(u_i, \epsilon) \\
u_i &= \alpha_{S[i]} + \beta_{1S[i]} Y_i \\
\alpha_j &= \bar{\alpha} + (z \times \sigma_\alpha) \\
\beta_{1j} &= \bar{\beta}_1 + (v \times \sigma_{\beta_1}) \\
z, v &\sim \mathcal{N}(0, 1) \\
\epsilon &\sim \mathrm{Exponential}(5) \\
\bar{\alpha}, \bar{\beta}_1 &\sim \mathcal{N}(0, 1.5) \\
\sigma_\alpha, \sigma_{\beta_1} &\sim \mathrm{Exponential}(3) \\
w_i &=
\begin{cases}
0 & \leftrightarrow (M_i = 0) \\
1 & \leftrightarrow (M_i > 0)
\end{cases}
\end{aligned}
\]

\paragraph{Model 5}\label{model-5}

\[
\begin{aligned}
w_i &\sim \mathrm{Bernoulli}(p_i) \\
\text{logit}(p_i) &= \theta + \delta_1 Y_i + \delta_2 \mathrm{PC1}_i \\
\theta, \delta_1, \delta_2 &\sim \mathcal{N}(0, 1.5) \\
(M_i \mid w_i = 1) &\sim \mathrm{LogNormal}(u_i, \epsilon) \\
u_i &= \alpha_{S[i]} + \beta_1 Y_i + \beta_2 \mathrm{PC1}_i + \beta_3 \mathrm{PC2}_i \\
\alpha_j &= \bar{\alpha} + (z \times \sigma_\alpha) \\
z &\sim \mathcal{N}(0, 1) \\
\epsilon &\sim \mathrm{Exponential}(5) \\
\bar{\alpha} &\sim \mathcal{N}(0, 1.5) \\
\sigma_\alpha &\sim \mathrm{Exponential}(3) \\
\beta_1, \beta_2, \beta_3 &\sim \mathcal{N}(0, 1.5) \\
w_i &=
\begin{cases}
0 & \leftrightarrow (M_i = 0) \\
1 & \leftrightarrow (M_i > 0)
\end{cases}
\end{aligned}
\]

\paragraph{Model 6}\label{model-6}

\[
\begin{aligned}
w_i &\sim \mathrm{Bernoulli}(p_i) \\
\text{logit}(p_i) &= \theta + \delta_1 Y_i \\
\theta, \delta_1 &\sim \mathcal{N}(0, 1.5) \\
(M_i \mid w_i = 1) &\sim \mathrm{LogNormal}(u_i, \epsilon) \\
u_i &= \alpha_{S[i]} + \beta_1 Y_i + \beta_2 \mathrm{PC1}_i \\
\alpha_j &= \bar{\alpha} + (z \times \sigma_\alpha) \\
z &\sim \mathcal{N}(0, 1) \\
\epsilon &\sim \mathrm{Exponential}(5) \\
\bar{\alpha} &\sim \mathcal{N}(0, 1.5) \\
\sigma_\alpha &\sim \mathrm{Exponential}(3) \\
\beta_1, \beta_2 &\sim \mathcal{N}(0, 1.5) \\
w_i &=
\begin{cases}
0 & \leftrightarrow (M_i = 0) \\
1 & \leftrightarrow (M_i > 0)
\end{cases}
\end{aligned}
\]

\paragraph{Model 7}\label{model-7}

\[
\begin{aligned}
w_i &\sim \mathrm{Bernoulli}(p_i) \\
\text{logit}(p_i) &= \theta + \delta_1 Y_i \\
\theta, \delta_1 &\sim \mathcal{N}(0, 1.5) \\
(M_i \mid w_i = 1) &\sim \mathrm{LogNormal}(u_i, \epsilon) \\
u_i &= \alpha_{S[i]} + \beta_1 Y_i \\
\alpha_j &= \bar{\alpha} + (z \times \sigma_\alpha) \\
z &\sim \mathcal{N}(0, 1) \\
\epsilon &\sim \mathrm{Exponential}(5) \\
\bar{\alpha} &\sim \mathcal{N}(0, 1.5) \\
\sigma_\alpha &\sim \mathrm{Exponential}(3) \\
\beta_1 &\sim \mathcal{N}(0, 1.5) \\
w_i &=
\begin{cases}
0 & \leftrightarrow (M_i = 0) \\
1 & \leftrightarrow (M_i > 0)
\end{cases}
\end{aligned}
\]

\paragraph{Model 8}\label{model-8}

\[
\begin{aligned}
w_i &\sim \mathrm{Bernoulli}(p_i) \\
\text{logit}(p_i) &= \theta \\
\theta &\sim \mathcal{N}(0, 1.5) \\
(M_i \mid w_i = 1) &\sim \mathrm{LogNormal}(u_i, \epsilon) \\
u_i &= \alpha \\
\alpha &\sim \mathcal{N}(0, 1.5) \\
\epsilon &\sim \mathrm{Exponential}(5) \\
w_i &=
\begin{cases}
0 & \leftrightarrow (M_i = 0) \\
1 & \leftrightarrow (M_i > 0)
\end{cases}
\end{aligned}
\]

The seven models along with the intercept-only null model (Model 8) were
ranked using the leave-one-out cross validation score which we estimated
using Pareto Smoothed Importance Sampling (PSIS) implemented using the
loo() function from the loo package (Vehtari et al. 2024) in R v4.1.214.
Model 1 had the best leave-one-out cross validation score (Table
S~\ref{tbl-s3}). Using 1000 samples from the posterior distribution of
Model 1, we imputed the dry biomass of missing monoculture values,
leaving us with 1000 separate estimates of the dry biomass for each
missing monoculture.

\begin{figure}

\centering{

\includegraphics{figures/app_4_fig_s17.png}

}

\caption{\label{fig-s17}Observed monoculture biomass and the predicted
monoculture biomass (mean within \(PI_{90\%}\)) based on the model fit
to the data with the lowest leave-one-out cross validation score for the
five different OTUs (colours) in the nine different clusters (plots).
The red, dashed line is the 1:1 line.}

\end{figure}%

We assessed the quality of these imputations obtained from the model
with the lowest PSIS-LOO score by examining the observed versus imputed
monoculture biomass values using the data that the model was fit to.
This showed that the model generally fit the data well and that most of
the observations were close to the 1:1 line (Figure S~\ref{fig-s17}). In
line with this, on average, a draw from the posterior distribution
predicted 8.7\% of the data to be zero whilst 8.4\% of the data was zero
in the observed monoculture dry biomass data. In addition, we compared
the distribution of the observed monoculture dry biomass values with the
distribution of imputed monoculture biomass values from the models for
each OTU. We found that the predictions overlapped considerably with the
observed dry biomass values (Figure S~\ref{fig-s18}). This gave us
further confidence in our imputations.

We also tested whether the values of the predictor variables (Y, PC1 and
PC2 of the monocultures that we imputed) were substantially beyond the
ranges of the predictor variables Y, PC1 and PC2 in the data used to fit
the model (i.e.~whether we were extrapolating beyond the range of data
used to fit the model when performing the imputations). To do this we
used the multivariate environmental similarity surface (MESS) index
(Zurell, Elith, and Schröder 2012). The MESS index splits each variable
from the observed data (in this case, Y, PC1 and PC2) into N bins. This
means that each observed data point has a combination of bin values
(e.g.~data point 1 for Y, PC1 and PC2: bin 1, bin 3 and bin 8). Any bin
combination (e.g.~1, 3, 8) that is not in the response variable data for
which predictions are made is an extrapolation (rather than an
interpolation). Using 20 bins, only 5.2\% of imputed monocultures had
values of Y, PC1 and PC2 that were outside the observed data bin
combinations. Using 10 bins, this dropped to 1.5\%. Thus, in between
94.8\% and 98.5\% of cases (depending on the number of bins chosen), the
predicted imputed monocultures were based on interpolating from our
model and not extrapolating.

\begin{figure}

\centering{

\includegraphics{figures/app_4_fig_s18.png}

}

\caption{\label{fig-s18}Distribution of the observed monoculture dry
biomass (g) data (red) for the five OTUs: Barn (N = 87), Bryo (N = 81),
Asci (N = 6), Hydro (N = 93) and Ciona (N = 20) compared to the imputed
monoculture dry biomass (g) for 25 random samples from the posterior
distribution (each grey line is a separate sample from the posterior
distribution).}

\end{figure}%

\newpage{}

\subsection{Reference list}\label{reference-list}

\phantomsection\label{refs}
\begin{CSLReferences}{1}{0}
\bibitem[\citeproctext]{ref-berntsson2003}
Berntsson, Kent M, and Per R and Jonsson. 2003. {``Temporal and Spatial
Patterns in Recruitment and Succession of a Temperate Marine Fouling
Assemblage: A Comparison of Static Panels and Boat Hulls During the
Boating Season.''} \emph{Biofouling} 19 (3): 187--95.
\url{https://doi.org/10.1080/08927014.2003.10382981}.

\bibitem[\citeproctext]{ref-fox2005}
Fox, Jeremy W. 2005. {``Interpreting the {`}Selection Effect{'} of
Biodiversity on Ecosystem Function.''} \emph{Ecology Letters} 8 (8):
846--56. \url{https://doi.org/10.1111/j.1461-0248.2005.00795.x}.

\bibitem[\citeproctext]{ref-greeve2023}
Greeve, Youk, Per Bergström, Åsa Strand, and Mats Lindegarth. 2023.
{``Estimating and Scaling-up Biomass and Abundance of Epi- and Infaunal
Bivalves in a Swedish Archipelago Region: Implications for Ecological
Functions and Ecosystem Services.''} \emph{Frontiers in Marine Science}
10 (January). \url{https://doi.org/10.3389/fmars.2023.1105999}.

\bibitem[\citeproctext]{ref-isbell2018}
Isbell, Forest, Jane Cowles, Laura E. Dee, Michel Loreau, Peter B.
Reich, Andrew Gonzalez, Andy Hector, and Bernhard Schmid. 2018.
{``Quantifying Effects of Biodiversity on Ecosystem Functioning Across
Times and Places.''} \emph{Ecology Letters} 21 (6): 763--78.
\url{https://doi.org/10.1111/ele.12928}.

\bibitem[\citeproctext]{ref-loreau2001}
Loreau, Michel, and Andy Hector. 2001. {``Partitioning Selection and
Complementarity in Biodiversity Experiments.''} \emph{Nature} 412
(6842): 72--76. \url{https://doi.org/10.1038/35083573}.

\bibitem[\citeproctext]{ref-loreau2003}
Loreau, Michel, Nicolas Mouquet, and Andrew Gonzalez. 2003.
{``Biodiversity as Spatial Insurance in Heterogeneous Landscapes.''}
\emph{Proceedings of the National Academy of Sciences} 100 (22):
12765--70. \url{https://doi.org/10.1073/pnas.2235465100}.

\bibitem[\citeproctext]{ref-loreau2002}
Loreau, Michel, Shahid Naeem, and Pablo Inchausti. 2002.
\emph{Biodiversity and Ecosystem Functioning: Synthesis and
Perspectives}. Oxford University Press.

\bibitem[\citeproctext]{ref-mcelreath2018}
McElreath, Richard. 2018. \emph{Statistical Rethinking: A Bayesian
Course with Examples in r and Stan}. New York: Chapman; Hall/CRC.
\url{https://doi.org/10.1201/9781315372495}.

\bibitem[\citeproctext]{ref-muukkonen2006}
Muukkonen, Petteri, Raisa Mäkipää, Raija Laiho, Kari Minkkinen, Harri
Vasander, and Leena Finér. 2006. {``Relationship Between Biomass and
Percentage Cover in Understorey Vegetation of Boreal Coniferous
Forests.''} \emph{Silva Fennica} 40 (2).
\url{https://doi.org/10.14214/sf.340}.

\bibitem[\citeproctext]{ref-vegan}
Oksanen, Jari, Gavin L. Simpson, F. Guillaume Blanchet, Roeland Kindt,
Pierre Legendre, Peter R. Minchin, R. B. O'Hara, et al. 2024. {``Vegan:
Community Ecology Package.''}
\url{https://CRAN.R-project.org/package=vegan}.

\bibitem[\citeproctext]{ref-porter2000}
Porter, Elka T., Lawrence P. Sanford, and Steven E. Suttles. 2000.
{``Gypsum Dissolution Is Not a Universal Integrator of {`}Water
Motion{'}.''} \emph{Limnology and Oceanography} 45 (1): 145--58.
\url{https://doi.org/10.4319/lo.2000.45.1.0145}.

\bibitem[\citeproctext]{ref-rueden2017}
Rueden, Curtis T., Johannes Schindelin, Mark C. Hiner, Barry E. DeZonia,
Alison E. Walter, Ellen T. Arena, and Kevin W. Eliceiri. 2017.
{``ImageJ2: ImageJ for the Next Generation of Scientific Image Data.''}
\emph{BMC Bioinformatics} 18 (1): 529.
\url{https://doi.org/10.1186/s12859-017-1934-z}.

\bibitem[\citeproctext]{ref-rstan}
Stan Development Team. 2024. {``{\textbraceleft}RStan{\textbraceright}:
The {\textbraceleft}r{\textbraceright} Interface to
{\textbraceleft}Stan{\textbraceright}.''} \url{https://mc-stan.org/}.

\bibitem[\citeproctext]{ref-thompson2020}
Thompson, Patrick L., Laura Melissa Guzman, Luc De Meester, Zsófia
Horváth, Robert Ptacnik, Bram Vanschoenwinkel, Duarte S. Viana, and
Jonathan M. Chase. 2020. {``A Process-Based Metacommunity Framework
Linking Local and Regional Scale Community Ecology.''} \emph{Ecology
Letters} 23 (9): 1314--29. \url{https://doi.org/10.1111/ele.13568}.

\bibitem[\citeproctext]{ref-loo}
Vehtari, Aki, Jonah Gabry, Måns Magnusson, Yuling Yao, Paul-Christian
Bürkner, Topi Paananen, and Andrew Gelman. 2024. {``Loo: Efficient
Leave-One-Out Cross-Validation and WAIC for Bayesian Models.''}
\url{https://mc-stan.org/loo/}.

\bibitem[\citeproctext]{ref-vellend2010}
Vellend, Mark. 2010. {``Conceptual Synthesis in Community Ecology.''}
\emph{The Quarterly Review of Biology} 85 (2): 183--206.
\url{https://doi.org/10.1086/652373}.

\bibitem[\citeproctext]{ref-yachi1999}
Yachi, Shigeo, and Michel Loreau. 1999. {``Biodiversity and Ecosystem
Productivity in a Fluctuating Environment: The Insurance Hypothesis.''}
\emph{Proceedings of the National Academy of Sciences} 96 (4): 1463--68.
\url{https://doi.org/10.1073/pnas.96.4.1463}.

\bibitem[\citeproctext]{ref-zurell2012}
Zurell, Damaris, Jane Elith, and Boris Schröder. 2012. {``Predicting to
New Environments: Tools for Visualizing Model Behaviour and Impacts on
Mapped Distributions.''} \emph{Diversity and Distributions} 18 (6):
628--34. \url{https://doi.org/10.1111/j.1472-4642.2012.00887.x}.

\end{CSLReferences}




\end{document}
